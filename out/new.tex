\documentclass[12pt,onecolumn]{IEEEtran} % single-column, 12pt

% --- Essential packages ---
\usepackage[T1]{fontenc}
\usepackage[utf8]{inputenc}
\usepackage{lmodern} % Better font support
\usepackage{amsmath,amssymb,amsfonts}
\usepackage{algorithmic}
\usepackage{graphicx}
\usepackage{xcolor}
\usepackage{textcomp}
\usepackage{microtype}

% --- Page layout ---
\usepackage[margin=1in]{geometry}

% --- Font & spacing ---
\usepackage{newtxtext,newtxmath}
\usepackage{setspace}
\AtBeginDocument{\setstretch{1.5}}  % 1.5 line spacing
\usepackage{footmisc}
\setlength{\footnotesep}{1.5\baselineskip} % Adjust footnote spacing

% --- Table packages ---
\usepackage{array}
\usepackage{tabularx}
\usepackage{booktabs} % you already use \toprule/\midrule/\bottomrule
\usepackage{adjustbox}
\usepackage{ragged2e}
\usepackage{float}

% Custom column types
\newcolumntype{L}[1]{>{\raggedright\arraybackslash}p{#1}} % fixed-width, ragged-right
\newcolumntype{Y}{>{\raggedright\arraybackslash}X}        % flexible-width, ragged-right
\newcolumntype{P}[1]{>{\RaggedRight\arraybackslash}p{#1}} % fixed-width ragged-right column
\newcommand{\fitToPage}[1]{\begin{adjustbox}{max width=\textwidth}#1\end{adjustbox}}

% Table formatting
\setlength{\tabcolsep}{6pt}  % Adjusts column spacing
\renewcommand{\arraystretch}{1.2} % Set to 1.5 if you want table rows also at 1.5

% --- Section formatting ---
\usepackage{titlesec}
\titleformat{\part}[display]
  {\normalfont\bfseries\fontsize{14pt}{16.8pt}\selectfont}  % 14pt, bold
  {\partname~\thepart}{0.5em}{}
\titleformat{\section}
  {\normalfont\bfseries\fontsize{14pt}{16.8pt}\selectfont}
  {\thesection}{1em}{}
\titleformat{\subsection}
  {\normalfont\bfseries\fontsize{12pt}{14pt}\selectfont}
  {\thesubsection}{1em}{}
\titleformat{\subsubsection}
  {\normalfont\itshape\fontsize{12pt}{14pt}\selectfont}
  {\thesubsubsection}{1em}{}

% Section spacing
\titlespacing*{\section}{0pt}{*2}{*1}
\titlespacing*{\subsection}{0pt}{*1.5}{*0.8}
\titlespacing*{\subsubsection}{0pt}{*1.2}{*0.6}

% --- Numbering format ---
\setcounter{secnumdepth}{3}
\setcounter{tocdepth}{3}

\makeatletter
\renewcommand\thesection{\arabic{section}}
\renewcommand\thesubsection{\thesection.\arabic{subsection}}
\renewcommand\thesubsubsection{\thesubsection.\arabic{subsubsection}}

% For table of contents
\renewcommand{\numberline}[1]{\hb@xt@\@tempdima{#1\hfil}}
\makeatother

% --- Captions ---
\usepackage{caption}
\captionsetup{font={small,stretch=1.5}}  % Captions with 1.5 line spacing and small font size

% --- Hyperlinks & references ---
\usepackage[hyphens]{url}
\Urlmuskip=0mu plus 1mu
\usepackage[hidelinks]{hyperref}
\hypersetup{
    pdftitle={Blockchain-enabled IoT Framework for Smart Agriculture},
    pdfauthor={},
    pdfpagemode=UseOutlines,
}
\usepackage{cite}

% --- Table of Contents formatting with tocloft package ---
\usepackage{tocloft}
\setlength{\cftsecindent}{0em}     % No indent for sections
\setlength{\cftsubsecindent}{1.5em} % Indent for subsections
\setlength{\cftsubsubsecindent}{3em} % Indent for subsubsections
\setlength{\cftsecnumwidth}{2.5em}  % Adjust the number width for sections
\setlength{\cftsubsecnumwidth}{3.5em} % Adjust the number width for subsections
\setlength{\cftsubsubsecnumwidth}{4em} % Adjust the number width for subsubsections
\setlength{\cftbeforesecskip}{1em}   % Space before each section in ToC
\setlength{\cftbeforesubsecskip}{0.5em} % Space before each subsection
\setlength{\cftbeforesubsubsecskip}{0.5em} % Space before each subsubsection

% --- Document info ---
\title{Literature Review and Comparative Analysis}


\begin{document}

\maketitle

% Force TOC to start on a new page

\tableofcontents
\clearpage










%%%%%%%%%%%%%%%%%%%%%%%%%%%%%%%%%%%%%%%%%%%%%%%%%%%%%%%%%%%%
% PART I – Foundations and Context
%%%%%%%%%%%%%%%%%%%%%%%%%%%%%%%%%%%%%%%%%%%%%%%%%%%%%%%%%%%%
\part*{Part I – Foundations and Context}\addcontentsline{toc}{part}{Part I – Foundations and Context}

\section{Introduction}\label{sec:introduction}
The agricultural sector is undergoing a significant digital transformation driven by the convergence of emerging technologies including Internet of Things (IoT), blockchain, artificial intelligence, and cloud computing. Smart agriculture has evolved as an interdisciplinary field that integrates IoT sensor networks for continuous monitoring of environmental variables, soil properties, and crop health, enabling data-driven decision-making and precision farming \cite{dhanaraju2022smartfarminginternet, akhter2022precisionagricultureusing}. Modern agricultural operations generate extensive real-time sensor data that can overwhelm traditional data management systems, particularly when collected from multiple fields and diverse sensor types \cite{huang2025digitaltraceabilityin, irfan2025aniotdrivensmart}.

Despite the promising capabilities of IoT technologies in agriculture, traditional deployments face several limitations including restricted connectivity in remote areas, high energy consumption, limited scalability, security vulnerabilities, and lack of standardized protocols \cite{bayih2022utilizationofinternet, quy2022iotenabledsmartagriculture, jaliyagoda2023internetofthings}. Centralized cloud architectures present additional vulnerabilities such as single points of failure and unauthorized data manipulation, potentially compromising data integrity and operational reliability \cite{aliyu2023blockchainbasedsmartfarm}. The heterogeneous nature of agricultural IoT ecosystems, comprising devices from multiple manufacturers with proprietary communication protocols, further complicates cross-platform data integration and real-time response capabilities \cite{irfan2025aniotdrivensmart, abdurrohim2024blockchainbasedframeworkfor}.

Blockchain technology has emerged as a promising solution to these challenges, offering an immutable, decentralized ledger that establishes transparency and trust through consensus mechanisms and cryptographic safeguards. Research demonstrates that integrating blockchain's distributed ledger with IoT sensor networks enables trustworthy data management and automated decision-making in precision farming environments \cite{aliyu2023blockchainbasedsmartfarm, ellahi2023blockchainbasedframeworksfor}. This integration ensures traceability, secure data sharing, and tamper-resistance for continuously generated agricultural data \cite{daund2025designofan}.

However, significant challenges remain in implementing blockchain solutions for agricultural IoT applications. Traditional consensus protocols often exhibit substantial resource requirements and scalability constraints that prove prohibitive for resource-constrained IoT deployments \cite{akella2023asystematicreview, mwewa2024blockchaintechnologya}. The energy consumption associated with conventional consensus mechanisms presents particular challenges in agricultural settings where IoT devices typically operate on limited energy budgets \cite{ninsiima2025determinantsofsmallholder}. Additional usability challenges hinder practical adoption among farmers, who may lack digital literacy or access to robust IT infrastructures \cite{mwewa2024blockchaintechnologya}.

This report reviews recent literature from 2022 to 2025, categorizing research into seven thematic groups:
\begin{enumerate}
    \item Blockchain applications in smart agriculture
    \item IoT architectures in agriculture
    \item Consensus mechanisms and performance
    \item QoS issues in blockchain-IoT systems
    \item Security and privacy preservation mechanisms
    \item Energy efficiency considerations
    \item Usability and supply chain integration
\end{enumerate}

The report has three primary objectives:
\begin{enumerate}
    \item To provide a comprehensive synthesis of state-of-the-art research in blockchain-enabled IoT frameworks for smart agriculture
    \item To develop comparative analyses summarizing key findings, methodologies, and limitations across different research domains
    \item To critically evaluate how Hyperledger solutions and our proposed CRT-based parallel transaction model can address existing research gaps
\end{enumerate}

Our proposed solution introduces a novel CRT-based parallel transaction model that partitions blockchain processing into multiple parallel streams using mathematical properties of the Chinese Remainder Theorem. This model incorporates lightweight consensus algorithms, reputation-based schemes tailored for agricultural sensor data, and QoS mechanisms to dynamically prioritize critical sensor events \cite{thiruvenkatasamy2025anonlinetool, huang2025digitaltraceabilityin}. The approach aims to address computational challenges associated with high-volume blockchain transactions while resolving interoperability issues across heterogeneous devices, ultimately enabling real-time response capabilities essential for precision agriculture applications including irrigation management, pest detection, and supply chain validation.

The following sections present an enhanced literature review, comparative analyses, and a detailed mapping of identified limitations to potential solutions incorporating both Hyperledger frameworks and our CRT-based model.


\section{Background and Motivation}\label{sec:background}
Traditional blockchain systems rely heavily on consensus protocols that guarantee security and data integrity through high computational overhead, as seen in Proof of Work (PoW) \cite{ali2022blockchainenabledarchitecture}. However, for IoT applications, and smart agriculture in particular, the inherent limitations in processing power, memory, and energy necessitate consensus algorithms that are lightweight, scalable, and adaptive \cite{bryant2022keychallengesin, khan2022asurveyand}. Recent studies emphasize that emerging architectures for blockchain-enabled IoT have increasingly incorporated tailored consensus mechanisms such as selective consensus \cite{ali2022blockchainenabledarchitecture}, hierarchical models \cite{guo2022ahierarchicaland}, and reputation-based leader election \cite{morais2023surveyonintegration}. In addition, many works have addressed integrating edge computing alongside blockchain to offload heavy computations, thereby reducing consensus latency \cite{khan2022asurveyand, haque2024ascalableblockchain}. As smart agriculture demands near real-time data transmission and coordinated decision making, any proposed IoT blockchain framework must accommodate low latency, high throughput, and assured security.

\section{Blockchain in Smart Agriculture}\label{sec:blockchain}
Recent studies in smart agriculture explore blockchain's application for secure, immutable traceability across entire agricultural supply chains and for real-time crop monitoring. For instance, research by Ahmed Abubakar Aliyu and Jinshuo Liu \cite{aliyu2023blockchainbasedsmartfarm} proposed a blockchain-based smart farm security framework that integrates IoT sensors with both Ethereum and Hyperledger Fabric to perform remote crop monitoring and supply-chain traceability. Their methodology relies on smart contracts for automated enforcement of security rules, but the work is limited by scalability challenges and the small test sizes used in experiments. Similarly, another recent paper \cite{aliyu2023blockchainbasedsmartfarm} investigated blockchain-enabled IoT frameworks to secure real-time device data in smart agriculture environments, highlighting vulnerabilities and emphasizing continuous device health monitoring. Despite robust design, the proposed prototype does not yet offer a fully real-time responsive system and requires better integration with heterogeneous IoT hardware.

Moreover, studies such as the systematic review by Ellahi et al. \cite{ellahi2023blockchainbasedframeworksfor} emphasize that blockchain can augment food traceability and supply chain transparency by maintaining immutable audit trails, yet these proposals often face issues such as high transaction latency and high energy consumption on public blockchains. Other works \cite{sakthivel2024enhancingtransparencyand, sizan2505asecuredtriad} propose novel architectures that integrate sensor networks with blockchain layers, achieving high crop monitoring accuracy through machine learning algorithms; however, their experiments reveal unresolved challenges in scalability and the selection of consensus mechanisms optimized for resource-limited devices. In addition, research addressing blockchain for supply chain traceability \cite{vitaskos2024blockchainandinternet, tang2024assessingblockchainand} demonstrates end-to-end traceability models for various agri-products but highlights inherent limitations such as integration complexity with legacy infrastructures and the high cost of deploying full blockchain solutions in rural areas.

\begin{table}[H]
\caption{Summary of Papers on Blockchain in Smart Agriculture}
\label{table1}
\centering
\fitToPage{
\begingroup
\setlength{\tabcolsep}{4pt}
\renewcommand{\arraystretch}{1.15}
\footnotesize
\begin{tabularx}{\textwidth}{@{}P{0.18\textwidth}YYY@{}}
\toprule
\textbf{Paper} & \textbf{Key Findings} & \textbf{Methodologies} & \textbf{Limitations} \\
\midrule
Aliyu and Liu (2023) & Proposed smart farm security framework integrating IoT with blockchain for traceability and remote monitoring & Use of Ethereum \& Hyperledger smart contracts for automation; neural network-based classification for threat detection & Scalability challenges, limited test data sizes, energy constraints on IoT devices \\
\addlinespace
Aliyu and Liu (2023) & Developed blockchain-based IoT framework for secure device monitoring; introduced event-driven smart contracts & Combined IoT sensor data with blockchain via Ethereum simulations; continuity of device health monitoring & Lack of full real-time response, integration issues with heterogeneous sensor networks \\
\addlinespace
Ellahi et al. (2023) & Enhanced traceability and transparency in agri-food supply chains using blockchain and IoT integration & Deployed private blockchain platforms with distributed applications and edge computing; integration with IPFS & High transaction latency, cost inefficiencies, scalability challenges within large datasets \\
\addlinespace
Sakthivel et al. (2024) & Introduced Hyperledger-based architecture for precision agriculture, focusing on supply chain security and traceability & Robust registration phases, secured key exchange, and IoT sensor data integration with blockchain & Scalability, energy efficiency of consensus mechanisms, legal and policy adaptation required \\
\addlinespace
Sizan et al. (2025) & IoT + blockchain for crop forecasting using ML; high prediction accuracy & ML models with blockchain for sensor data integrity & Reliance on test networks, limited large-scale dataset evaluation, consensus not fully optimized for IoT \\
\bottomrule
\end{tabularx}
\endgroup
}
\end{table}

These works share similarities in that they use smart contracts to enforce traceability and security, but differences arise in the blockchain platforms used (e.g., Ethereum vs. Hyperledger) and the consensus approaches applied. Many of these approaches are hindered by high energy consumption and limited scalability when applied to resource-constrained agricultural environments \cite{aliyu2023blockchainbasedsmartfarm, ellahi2023blockchainbasedframeworksfor}.

Building on these blockchain-focused insights, the next section examines how the underlying IoT architectures support or constrain such deployments.


\section{IoT Architectures in Agriculture}\label{sec:iot-architectures}
IoT architectures in agriculture are designed to enhance the monitoring of soil conditions, crop growth, pest detection, and irrigation management through the deployment of sensor networks and edge/fog computing. Several recent studies have examined lightweight, scalable IoT designs that integrate with blockchain to improve data collection accuracy and ensure secure data storage. For example, research by Osmanoglu et al. (mentioned in \cite{ellahi2023blockchainbasedframeworksfor}) demonstrates that combining IoT sensor networks with edge computing and blockchain can overcome latency and bandwidth limitations associated with conventional cloud computing. Other works have proposed modular IoT frameworks that incorporate real-time data collection, device authentication, and decentralized processing. In addition, studies such as those by Tsang et al. \cite{tang2024assessingblockchainand} and Malik et al. (implied in \cite{ellahi2023blockchainbasedframeworksfor}) complement blockchain's traceability capabilities by employing distributed IoT sensor networks that enable the collection of environmental data crucial for precision agriculture.

Many approaches adopt a hierarchical, multi-layered architecture where the bottom layer comprises resource-constrained IoT devices, and higher layers involve more powerful edge gateways and cloud computing for data aggregation and analysis. However, limitations persist: sensor noise, limited battery capacity, data fragmentation, and issues related to the interoperability of heterogeneous IoT devices \cite{aliyu2023blockchainbasedsmartfarm, ali2022blockchainenabledarchitecture}. Some proposals further attempt to mitigate these limitations by integrating RFID systems, AR/VR technologies, or IoT gateway solutions, though they require significant tuning for rural contexts where infrastructural deficits are common \cite{tang2024assessingblockchainand}.

\begin{table}[H]
\caption{Summary of Papers on IoT Architectures in Agriculture}
\label{table2}
\centering
\fitToPage{
\begingroup
\setlength{\tabcolsep}{4pt}
\renewcommand{\arraystretch}{1.15}
\footnotesize
\begin{tabularx}{\textwidth}{@{}P{0.18\textwidth}YYY@{}}
\toprule
\textbf{Paper} & \textbf{Key Findings} & \textbf{Methodologies} & \textbf{Limitations} \\
\midrule
Ali and Sofi (2022) & Three-tier blockchain-enabled IoT architecture for secure agri data & IoT device, blockchain, and application layers; edge gateways for pre-processing & Choosing optimal consensus; reliance on edge nodes due to constraints \\
\addlinespace
Ellahi et al. (2023) & Blockchain + IoT improves traceability and reduces fraud & Distributed IoT sensors; privacy-preserving protocols & Scalability/cost limits on terminal devices \\
\addlinespace
Tang et al. (2024) & Lightweight blockchain on sensors for African supply chains & IoT modules + smart contracts; IPFS off-chain storage & Energy challenges; heterogeneous device integration complexity \\
\addlinespace
Vitaskos et al. (2024) & Better monitoring via IoT data + blockchain contracts & MQTT comms; node validation; automated threshold alerts & On-chain storage limits; limited real-time on public chains \\
\addlinespace
Sakthivel et al. (2024) & IoT + blockchain + AI for decisions/traceability & Advanced sensors + smart contracts + RL & Scalability; potentially high hardware/cloud costs \\
\bottomrule
\end{tabularx}
\endgroup
}
\end{table}


These studies are similar in their emphasis on using multi-tiered architectures to cope with bandwidth, latency, and computational constraints present in rural agricultural contexts. Nevertheless, many of their limitations such as hardware energy consumption and integration of heterogeneous sensor networks remain open challenges \cite{morais2023surveyonintegration, tang2024assessingblockchainand}.

Addressing these architectural challenges requires suitable consensus strategies, which are reviewed in the following section.

\section{Security, Privacy, and Reliability Considerations}\label{sec:spr}
% Parent container (no text unless you have an overview).
  \subsection{Security Aspects in Blockchain-Enabled Smart Agriculture}\label{subsec:security}
Security remains the foremost priority in deploying IoT systems in agriculture. Early research laid the foundation by showing that blockchain's inherent characteristics—decentralization, cryptographic hashing, and immutable ledger storage—can significantly improve data integrity and reduce vulnerabilities \cite{aliyu2023blockchainbasedsmartfarm}. For instance, Aliyu and Liu (2023) proposed a blockchain-based smart farm security framework which leverages immutable transaction records and smart contracts to enforce access controls and enable real-time monitoring of IoT sensor health. Their framework employs technologies such as Arduino sensor kits, AWS cloud services, and the Ethereum blockchain via the Rinkeby test network to trigger alerts when abnormal sensor readings are detected. The practical evaluations demonstrated that a decline in accepted blockchain transactions was a reliable indicator of potential security threats, such as poisoning attacks, thereby enhancing end-to-end transaction security and ensuring that tamper-resistant data were available as evidence for dispute resolution.

Other studies have focused on designing fault-tolerant architectures that incorporate consensus algorithms to manage potential single points of failure. For example, research on redactable blockchain-assisted secure data aggregation in fog-enabled Internet-of-Farming-Things (IoFT) systems introduced a three-tier architecture in which data from agricultural IoT devices are aggregated securely at the fog layer before being transmitted and stored in an immutable ledger in the cloud \cite{mishra2023redactableblockchainassistedsecure}. This approach enhances security by coupling source authentication with controlled data modification capabilities while maintaining high availability and protecting against collusion and false data-injection attacks. Additionally, several works have proposed lightweight blockchain implementations optimized for resource-constrained environments encountered in small-scale farms, although challenges such as increased energy consumption per transaction and higher computational overhead persist \cite{shahzad2025decentralizediotbasedarchitectures}.

Smart contracts play a central role in enforcing security policies in these architectures. They automate authentication, access control, and the validation of transactions across distributed networks. For example, blockchain platforms such as Hyperledger Fabric and Corda have been preferred for agricultural applications due to their permissioned networks, which enable controlled access and thereby reduce the risk of unauthorized data access \cite{aliyu2023blockchainbasedsmartfarm, soy2025blockchainintegrationin}. These platforms, however, have shown differences in scalability and transaction processing capabilities. Ethereum, despite its flexibility and strong support for smart contracts, is often challenged by scalability limitations that result in increased transaction costs and latency in real-time applications \cite{aliyu2023blockchainbasedsmartfarm}. Therefore, many implementations target permissioned blockchains that offer modularity, enabling security protocols specifically tailored for enterprise-level smart farming deployments.

Another noteworthy approach to security enhancement involves integrating advanced cryptographic techniques such as Elliptic Curve Cryptography (ECC), ring signature technology, and zero-knowledge proofs. Liu et al. demonstrated that the use of ECC not only improves data confidentiality during transmission but also aids robust access control mechanisms that are critical in mitigating insider threats and impersonation attacks in IoT networks. Moreover, lightweight anonymous authentication schemes and secure key management protocols have been developed to secure endpoints without relying on centralized certificate authorities, thus further decentralizing the risk and improving overall network resilience \cite{rai2024enhancingdatasecurity}.

Finally, cyberattack detection and intrusion prevention are enhanced through the incorporation of machine learning techniques. For example, the integration of Isolation Forest for unsupervised anomaly detection and Long Short-Term Memory (LSTM) networks for time-series threat detection has yielded detection rates exceeding 95\% accuracy \cite{daund2025designofan}. These approaches further augment blockchain-based security architectures by providing early warning signals to farmers via mobile applications and automated anomaly triggers embedded as part of smart contracts.

  \subsection{Privacy Preservation Mechanisms}\label{subsec:privacy}
Privacy concerns in smart agricultural IoT are multifaceted, involving unauthorized access to sensitive farm data, exposure of personal financial information, and risks associated with the misuse of longitudinal agricultural datasets. Recent literature underscores the importance of a privacy-centric framework that integrates secure data exchange, confidential analytics, and user anonymity without compromising the transparency of blockchain systems \cite{rahaman2024privacycentricaiand}.

One promising direction integrates blockchain technology with privacy-preserving machine learning techniques such as federated learning and differential privacy. In a recent study, researchers combined Hyperledger Fabric with federated learning to enable distributed model training, protecting raw data on local IoT devices while sharing only encrypted intermediate results \cite{daund2025designofan}. Differential privacy techniques are applied by adding calibrated noise to data contributions, ensuring that sensitive information remains anonymized while maintaining acceptable model accuracy. Such methods effectively counteract privacy leakage, a critical requirement given that centralized data aggregation in conventional cloud systems can be exploited by attackers or expose data through misconfigured access permissions.

Other works examine the integration of cryptographic techniques tailored to enhance privacy. Zero-Knowledge Proofs (ZKP), for instance, enable participants to validate the authenticity of transactions without revealing underlying data, striking an optimum balance between auditability and confidentiality \cite{soy2025blockchainintegrationin}. Similarly, ring signatures have been employed on elliptic curve platforms to ensure user anonymity in transaction validation, although these methods sometimes require trade-offs in terms of computational efficiency.

Blockchain-based privacy-preserving methods for supply chain management are also critical. In a food traceability context, privacy-preserving protocols ensure that proprietary information related to agricultural practices and product quality remains confidential while offering end-to-end transparency, thereby assuring stakeholders of data integrity. However, such systems face challenges related to the inherent conflict between blockchain transparency and individual privacy, often necessitating the use of hybrid models that combine public and private blockchain features \cite{soy2025blockchainintegrationin}.

Another aspect of privacy preservation is the security of data exchanged among heterogeneous IoT devices. For instance, a privacy-centric protocol designed for smart rural farm monitoring employs a three-phase scheme integrating symmetric and asymmetric key encryption, hash functions, and secure communication channels. This protocol has demonstrated resilience against identity guessing, impersonation, and man-in-the-middle attacks, while maintaining low computational overhead to suit resource-constrained rural environments \cite{rahaman2024privacycentricaiand}. Nonetheless, while these methods have excelled in controlled experimental settings, their scalability to large-scale deployments in the field remains an open question.

In summary, privacy preservation in blockchain-enabled IoT smart agriculture systems is achieved by combining federated learning with advanced encryption, ZKP, and anonymous authentication schemes. Although these methods have significantly improved data confidentiality and user privacy while enabling collaborative analytics, challenges persist in scaling these approaches with minimal additional resource consumption \cite{soy2025blockchainintegrationin}.

  \subsection{Reliability and Fault Tolerance}\label{subsec:reliability}
Reliability in smart agriculture IoT frameworks ensures that data collected from a diverse range of sensors remains accurate, timely, and available for decision-making despite the presence of dynamic network conditions and potential cyber threats. Efforts to create fault-tolerant blockchain architectures are motivated by the need to prevent data loss, mitigate single points of failure, and maintain operational continuity amid cyber-attacks \cite{aliyu2023blockchainbasedsmartfarm}.

Recent works have developed multi-tiered blockchain architectures that stratify processing into edge, fog, and cloud layers, each managed by designated "Data Handlers" to ensure effective data lifecycle management. One such architecture incorporates Local Agricultural Data Handlers at the edge for on-site sensor data capture, Peripheral Agri-Fog Data Handlers that facilitate low-latency transmission, and Cloud Agri-Data Handlers that process and analyze aggregated data using advanced optimization algorithms \cite{thiruvenkatasamy2025anonlinetool}. This hierarchical design distributes trust and processing load, increasing overall system fault tolerance and scalability.

Another approach leverages redactable blockchain techniques that allow controlled data modifications while upholding overall immutability. In fog-enabled Internet-of-Farming-Things (IoFT) systems designed for secure data aggregation, a redactable blockchain framework was proposed to selectively modify data while maintaining a record of all alterations. This method enhances reliability by ensuring that even if data inconsistencies are detected, they can be corrected without compromising the integrity of the overall ledger \cite{mishra2023redactableblockchainassistedsecure}. Nevertheless, the computational overhead associated with such encryption and redaction processes remains a limitation for real-world deployments.

Reliability is further exemplified in prototypes that integrate blockchain with cloud-based processing to provide real-time monitoring and automated responses. For instance, a significant prototype employs NodeMCU microcontrollers connected to a permissioned blockchain network whereby IoT sensors continuously report environmental parameters. In a case study applied to a cotton field, the system reduced water consumption by 35\% by triggering automated irrigation based on data recorded immutably on the blockchain \cite{shahzad2025decentralizediotbasedarchitectures}. This performance underscores the practical benefits of combining decentralization with real-time analytics. Key performance metrics in such implementations include throughput (measured in transactions per second), latency, and resource utilization, each of which must be balanced against the inherent overhead incurred by blockchain consensus protocols.

The literature also emphasizes the need for robust consensus mechanisms capable of operating in distributed agricultural environments characterized by intermittent connectivity and heterogeneous IoT devices. Conventional consensus methods such as Proof-of-Work (PoW) introduce high computational and energy costs, whereas alternatives such as Proof-of-Authority (PoA) and Practical Byzantine Fault Tolerance (PBFT) offer greater efficiency and scalability for permissioned networks \cite{soy2025blockchainintegrationin}. Still, selecting an appropriate consensus algorithm remains a trade-off between decentralization, energy efficiency, and overall throughput.

Despite these advances, several challenges persist. The integration of blockchain protocols with resource-constrained agricultural IoT devices may result in increased communication latency and processing delays, negatively impacting the real-time responsiveness necessary for critical decision-making processes. Furthermore, the deployment of blockchain-based fault-tolerant systems in diverse farming contexts still requires extensive field testing to validate performance under environmental and operational heterogeneity \cite{shahzad2025decentralizediotbasedarchitectures}.


%%%%%%%%%%%%%%%%%%%%%%%%%%%%%%%%%%%%%%%%%%%%%%%%%%%%%%%%%%%%
% PART II – Consensus Mechanisms and System Performance
%%%%%%%%%%%%%%%%%%%%%%%%%%%%%%%%%%%%%%%%%%%%%%%%%%%%%%%%%%%%
\part*{Part II – Consensus Mechanisms and System Performance}\addcontentsline{toc}{part}{Part II – Consensus Mechanisms and System Performance}

\section{Survey of Consensus Mechanisms in IoT}\label{sec:consensus-survey}
Recent literature presents a diverse range of consensus mechanisms adapted for IoT environments. These mechanisms can be grouped into several categories based on their design philosophy and suitability for resource-constrained devices:

  \subsection{Selective and Lightweight Consensus Algorithms}\label{subsec:consensus-light}
Ali and Sofi 2022 propose a novel blockchain architecture that uses a selective consensus approach to adapt consensus selection to the scale of the IoT network \cite{ali2022blockchainenabledarchitecture}. Their work particularly underscores that traditional algorithms like PoW are impractical for IoT and introduces a three-tier architecture that offloads heavy computations to edge gateways. Similarly, other studies have proposed lightweight protocols such as Proof of Authentication (PoAh) and modified Proof of Work variants in order to reduce energy consumption while retaining sufficient security \cite{haque2024ascalableblockchain, khan2022asurveyand}.

  \subsection{Hierarchical and Location-Aware Consensus Protocols}\label{subsec:consensus-hier}
Guo et al. 2022 introduced a hierarchical and location-aware consensus protocol, known as LH-Raft, tailored for IoT-blockchain applications \cite{guo2022ahierarchicaland}. By grouping IoT nodes based on physical proximity and reputation, LH-Raft reduces communication costs and latency while offering scalability for large IoT networks. This hierarchical approach is especially appealing in distributed environments such as agriculture, where devices may be spread over large areas and need localized consensus.

  \subsection{DAG-based and Hybrid Consensus Models}\label{subsec:consensus-dag}
Another trend involves moving away from conventional blockchain data structures to alternative models such as Directed Acyclic Graphs (DAGs) to tackle scalability and storage constraints \cite{khan2022asurveyand, bryant2022keychallengesin}. DAG-based frameworks like IOTA's Tangle support parallel transaction validation, which reduces bottlenecks and enhances throughput. In addition, several proposals advocate for hybrid consensus mechanisms that combine aspects of PoW, Proof of Stake (PoS), and Byzantine Fault Tolerance (BFT) protocols \cite{guru2023asurveyon, khan2022asurveyand}. Hybrid models aim to balance decentralization, security, and resource utilization, although many still face limitations such as residual energy demands and complex consensus rule sets.

  \subsection{Reputation and Credit-Based Models}\label{subsec:consensus-rep}
There is significant literature on consensus where nodes are evaluated based on reputation or credit mechanisms \cite{morais2023surveyonintegration, khan2022asurveyand}. For instance, some models use voting schemes that incorporate node reputation, achieving weighted voting to finalize blocks in a manner that is both scalable and secure. Nonetheless, many such models note limitations in scalability when the number of nodes increases or under dynamic IoT conditions, and in some instances, they do not completely mitigate centralization risks.
  \subsection{Integration of Machine Learning Techniques}\label{subsec:consensus-ml}
Recent work has increasingly merged consensus mechanism development with machine learning (ML) in order to adapt consensus strategies in real time based on current network conditions \cite{alam2023anoverviewof, haque2024ascalableblockchain, guru2023asurveyon}. ML-driven consensus models have demonstrated the potential to dynamically adjust parameters such as leader election probabilities and transaction batching, thereby improving throughput and reducing latency. Despite promising improvements, these approaches also face challenges regarding the computational overhead of ML algorithms when deployed on IoT edge devices, where resource constraints are pronounced.

\section{Consensus Mechanisms \& Performance in IoT-Blockchain Systems}\label{sec:consensus-performance}
Consensus mechanisms are critical for maintaining data integrity and trust in decentralized blockchain-based IoT frameworks. Several recent studies have examined alternative consensus algorithms to conventional Proof-of-Work (PoW), which is unsuitable for resource-constrained IoT environments due to its high energy consumption. Instead, lightweight consensus protocols such as Proof-of-Stake (PoS), Delegated Proof-of-Stake (DPoS), Practical Byzantine Fault Tolerance (PBFT), and hybrid solutions have been proposed. For instance, a study outlined in \cite{ali2022blockchainenabledarchitecture} surveyed various consensus algorithms tailored for agricultural IoT networks, comparing PBFT's low latency with DPoS's scalability benefits. Other works \cite{morais2023surveyonintegration, khan2022ablockchainand} have introduced selective consensus mechanisms that dynamically choose a protocol based on network size and required throughput. These approaches significantly reduce computational overhead and energy consumption while maintaining decentralized trust.

Furthermore, some proposals integrate consensus mechanism evaluation with smart contract automation, enabling secure payments, crop certifications, and triggering alerts based on sensor inputs \cite{saha2022blockchainchangingthe, khan2022ablockchainand}. Despite these advances, limitations persist. Many studies report that while lightweight consensus models exhibit lower energy consumption, they often sacrifice throughput or security under high network loads \cite{saha2022blockchainchangingthe, ali2022blockchainenabledarchitecture}. There is also limited real-world testing, and much of the evaluation remains in simulation environments \cite{khan2022ablockchainand}.

\begin{table}[H]
\caption{Summary of Papers on Consensus Mechanisms \& Performance}
\label{table3}
\centering
\fitToPage{
\begingroup
\setlength{\tabcolsep}{4pt}
\renewcommand{\arraystretch}{1.15}
\footnotesize
\begin{tabularx}{\textwidth}{@{}P{0.18\textwidth}YYY@{}}
\toprule
\textbf{Paper} & \textbf{Key Findings} & \textbf{Methodologies} & \textbf{Limitations} \\
\midrule
Ali and Sofi (2022) & PBFT/DPoS/Ripple-like approaches vs. latency/scalability & Comparative analysis; simulations of overhead/fault tolerance & Some algorithms costly; balancing security/efficiency is hard \\
\addlinespace
de Morais et al. (2023) & Layered/dual-chain with smart contracts for traceability & Selective consensus architectures & Decentralization and overhead trade-offs persist \\
\addlinespace
Khan et al. (2022) & Selective consensus for IoT traceability & Blockchain + cloud/fog for throughput & Limited empirical testing; dynamic adaptation challenges \\
\addlinespace
Sakthivel et al. (2024) & Adaptive consensus with IoT sensors & Lightweight consensus + event-triggered contracts & Scalability in large networks; insufficient real-world validation \\
\addlinespace
Ellahi et al. (2023) & Survey of IoT–blockchain consensus (latency/energy focus) & Review + PoC prototypes & Lacking adversary-tolerance and field deployments \\
\bottomrule
\end{tabularx}
\endgroup
}
\end{table}



These papers share a common goal of optimizing blockchain consensus for resource-constrained environments, yet they differ in the specific algorithms they propose and in the extent of experimental validation. Many works suggest dynamic or selective consensus protocols, but further real-world testing is needed to assess their performance under various agricultural conditions \cite{ali2022blockchainenabledarchitecture, sakthivel2024enhancingtransparencyand}.

Because consensus design directly influences service quality, the subsequent section analyzes QoS considerations in blockchain-IoT deployments.

\section{Quality of Service (QoS) in Blockchain-IoT Systems}\label{sec:qos}
QoS parameters such as latency, throughput, and reliability are essential for ensuring that blockchain-enabled IoT frameworks in agriculture can support real-time monitoring and decision-making. Recent research has focused on mitigating delays and performance bottlenecks inherent in blockchain operations while maintaining secure data transmission over IoT networks. For example, research described in \cite{ellahi2023blockchainbasedframeworksfor} and \cite{ellahi2023blockchainbasedframeworksfor} emphasizes the development of off-chain storage solutions and smart contract optimizations to reduce latency and enhance throughput in agricultural supply chains. Other studies \cite{khan2022ablockchainand, sizan2505asecuredtriad} combine queuing theory with blockchain processing to optimize system performance and ensure that critical data is processed in near real time. A key contribution of these works is the demonstration that using edge and fog computing can improve the overall QoS by localizing data processing before storing verified data on the blockchain.

However, several limitations remain. Researchers often report high latency due to blockchain transaction confirmation times, especially on public networks, and the energy overhead of consensus processes can further compromise performance \cite{saha2022blockchainchangingthe, ali2022blockchainenabledarchitecture}. In addition, the integration of QoS methodologies with blockchain systems remains under-explored in heterogeneous environments, where different types of IoT devices and communication protocols must coexist \cite{sakthivel2024enhancingtransparencyand, sajja2023towardsapplicabilityof}.
  \subsection{Consensus–QoS Interactions for Smart Agriculture IoT}\label{subsec:qos-interactions}

  
\label{sec:consensus-qos}

\paragraph{QoS metrics.}
We use the following metrics:
\begin{enumerate}
    \item Latency $L$ (end-to-end);
    \item Jitter $J=\sqrt{\mathrm{Var}[D]}$ for delay $D$;
    \item Reliability $R=\Pr\{D\le D_{\max}\}$;
    \item Throughput $T$ (tx/s);
    \item Availability $A$ (uptime fraction meeting SLOs).
\end{enumerate}

\paragraph{Protocol models.}
\begin{enumerate}
    \item \textbf{PBFT}: with $n=3f+1$ replicas tolerates $f$ Byzantine faults; commit requires pre-prepare/prepare/commit rounds with $O(n^2)$ messages. A latency proxy is
    \[
    C_{\mathrm{PBFT}} \approx r\cdot \mathrm{RTT} + \tau_{\mathrm{sig}} + \tau_{\mathrm{batch}},\quad r\approx 3,
    \]
    where batching improves throughput but raises per-batch delay.

    \item \textbf{RAFT}: (crash-fault tolerant) elects a leader and commits on majority ($n=2f+1$). Latency is dominated by leader→followers replication and fsync:
    \[
    C_{\mathrm{RAFT}} \approx \mathrm{RTT}_{\mathrm{leader}} + \tau_{\mathrm{fsync}} + \tau_{\mathrm{batch}}.
    \]

    \item \textbf{PoS committees}: (e.g., VRF-selected) reduce validator sets to $k\ll n$, cutting quadratic costs. Latency is approximated as
    \[
    C_{\mathrm{PoS}}\approx \mathrm{RTT}_{\text{committee}} + \tau_{\mathrm{agg}},
    \]
    where $\tau_{\mathrm{agg}}$ accounts for aggregation/threshold signatures.
\end{enumerate}

\paragraph{Queueing/resource allocation.}
Model each shard as a service station with service time
\[
S_i = \tau_{\mathrm{exec},i} + C_i(B_i) + \tau_{\mathrm{net},i},\quad \mu_i = 1/\mathbb{E}[S_i],
\]
where $B_i$ is the batch size. For M/M/1,
\[
W_i=1/(\mu_i-\lambda_i).
\]
\textit{Batching trade-off:} increasing $B_i$ reduces per-tx overhead in $C_i$ (higher $\mu_i$) but inflates $W_i$ at low load. Choose $B_i^\star$ by minimizing $L_i(W_i+C_i)$ subject to SLOs (deadline $D_{\max}$, jitter cap $J_{\max}$).

\paragraph{Reliability vs. deadline.}
Under the M/M/1 approximation,
\[
R_i(d)=\Pr\{W_i\le d\}\approx 1-\rho_i e^{-(\mu_i-\lambda_i)d},\quad \rho_i=\lambda_i/\mu_i.
\]
Meeting $R\ge R^\star$ at deadline $d=D_{\max}$ yields capacity constraints on $(\lambda_i,\mu_i)$ that inform consensus choice and batching.

\paragraph{Mapping to agricultural workloads.}
\begin{enumerate}
    \item \textit{Control loops / actuation}: strict latency/jitter $\Rightarrow$ smaller $B_i$, faster consensus (committee PoS or tuned RAFT), higher $k$ (more shards).
    \item \textit{Telemetry / logs}: relaxed latency $\Rightarrow$ larger $B_i$, PBFT for integrity or energy-lean PoS committees.
    \item \textit{Payments / settlement}: prioritize finality; PBFT or PoS committees with strong finality, moderate $B_i$.
\end{enumerate}

\noindent\textit{Design rule:} pick consensus to shape $(\mu_i,C_i)$ so that $L_{\mathrm{e2e}}$ and $R$ from §\ref{sec:crt-perf} satisfy per-class SLOs; adjust $k$ and $B_i$ accordingly.


\begin{table}[H]
\caption{Summary of Papers on QoS in Blockchain-IoT Systems}
\label{table4}
\centering
\fitToPage{
\begingroup
\setlength{\tabcolsep}{4pt}
\renewcommand{\arraystretch}{1.15}
\footnotesize
\begin{tabularx}{\textwidth}{@{}P{0.18\textwidth}YYY@{}}
\toprule
\textbf{Paper} & \textbf{Key Findings} & \textbf{Methodologies} & \textbf{Limitations} \\
\midrule
Ellahi et al. (2023) & QoS gains via off-chain storage + smart contract tuning & On-chain/off-chain mix with flexible contracts & Scalability issues; IoT node energy not fully addressed \\
\addlinespace
Ellahi et al. (2023) & End-to-end traceability with better latency/throughput & Smart contracts + auth + cloud/fog load mgmt & Congestion under high loads; limited field deployments \\
\addlinespace
Khan et al. (2022) & QoS-aware IoT + blockchain for irrigation/monitoring & Lightweight consensus; queuing models & Limited real-world variation; model complexity \\
\addlinespace
Sakthivel et al. (2024) & Security–QoS balance in supply chains & Hybrid cloud/edge/DLT design & High setup cost; some QoS (e.g., jitter) not optimized \\
\addlinespace
Tang et al. (2024) & QoS via decentralized consensus in agri testbeds & Simulation of latency/throughput/resource use & Limited scalability tests; legacy integration issues \\
\bottomrule
\end{tabularx}
\endgroup
}
\end{table}



These studies are similar in their aim to combine blockchain's security and immutable architecture with advanced QoS techniques such as off-chain storage and smart contract optimization. Differences in the approaches lie in the specific methodologies employed for latency reduction and throughput enhancement, with many studies calling for further experimental validation in natural settings \cite{ellahi2023blockchainbasedframeworksfor, ellahi2023blockchainbasedframeworksfor}.

The insights from these QoS studies set the stage for a broader critical analysis and mapping of research gaps to Hyperledger capabilities.



%%%%%%%%%%%%%%%%%%%%%%%%%%%%%%%%%%%%%%%%%%%%%%%%%%%%%%%%%%%%
% PART III – Literature Review and Comparative Analysis
%%%%%%%%%%%%%%%%%%%%%%%%%%%%%%%%%%%%%%%%%%%%%%%%%%%%%%%%%%%%
\part*{Part III – Literature Review and Comparative Analysis}\addcontentsline{toc}{part}{Part III – Literature Review and Comparative Analysis}

\section{Literature Review}\label{sec:lit-review}
Recent scholarly contributions have extensively explored IoT architectures in agriculture, covering topics from network topology design and sensor deployment strategies to data processing methodologies and security frameworks. A thorough review of studies published between 2022 and 2025 reveals several thematic clusters: IoT-enabled precision agriculture systems, sensor networks in smallholder farms, smart greenhouse monitoring solutions, and integrated IoT–AI frameworks.

% Keep all their subsections below (copy content into each).
  \subsection{IoT-enabled Precision Agriculture and Farm Monitoring Systems}\label{subsec:lit-precision}
Researchers have investigated diverse IoT architectures that facilitate remote monitoring and decision-making in precision agriculture. For instance, studies have demonstrated the integration of wireless sensor networks (WSNs) with machine learning algorithms to predict crop diseases and optimize irrigation schedules \cite{dhanaraju2022smartfarminginternet, akhter2022precisionagricultureusing}. In one such work, sensor nodes deployed across apple orchards captured environmental metrics including temperature, humidity, and soil moisture, while cloud and fog computing platforms processed these data inputs in near real-time to detect apple scab disease \cite{akhter2022precisionagricultureusing}. Simulation models, such as those using the COOJA simulator and random waypoint mobility models, have been adopted to evaluate network performance metrics in both stationary (olive orchards) and mobile (animal farms) settings \cite{jaliyagoda2023internetofthings, atalla2023iotenabledprecisionagriculture}. Key findings from these studies include enhanced energy efficiency, reduced latency, and increased network throughput when employing optimized routing protocols like RPL and when leveraging hybrid edge–cloud architectures \cite{atalla2023iotenabledprecisionagriculture, atalla2023iotenabledprecisionagriculture}. However, significant limitations have been noted; chief among these are the reliance on simulation-based validations that do not fully capture real-world environmental dynamics, constraints in deployment scalability, and limited user-friendliness for farmers, particularly smallholders \cite{bayih2022utilizationofinternet, atalla2023iotenabledprecisionagriculture}.
  \subsection{IoT Sensor Networks in Smallholder and Precision Agriculture}\label{subsec:lit-sensornets}
For smallholder farms, where resources are limited, several studies underscore the need for low-cost, low-power IoT solutions that can be deployed at scale \cite{abunadi2022trafficawaresecuredcooperative, bayih2022utilizationofinternet}. Investigations into sensor network deployments typically focus on precision irrigation management through soil moisture monitoring, temperature, and nutrient levels. For example, research employing off-the-shelf devices like Arduino and Raspberry Pi has demonstrated that simple threshold-based sensor triggers can provide actionable data; however, these methods often underutilize the full potential of advanced data analytics and fail to integrate sophisticated machine learning techniques \cite{bayih2022utilizationofinternet, bayih2022utilizationofinternet}. Additionally, smallholder contexts emphasize cost-effectiveness and local production, which further constrain the inclusion of advanced sensor technologies and robust network infrastructures \cite{bayih2022utilizationofinternet, bayih2022utilizationofinternet}. Limitations in these studies frequently revolve around energy efficiency issues, sensor calibration errors, and a lack of standardized communication protocols that hinder interoperability across heterogeneous devices \cite{bayih2022utilizationofinternet, bayih2022utilizationofinternet}.
  \subsection{Smart Greenhouse and Controlled Environment Agriculture}\label{subsec:lit-greenhouse}
In the domain of greenhouse monitoring, researchers have developed IoT-based systems that integrate environmental sensors with cloud computing services to facilitate remote climate control \cite{viswanatha2022implementationofiot, simo2022smartagricultureiotbased}. Smart greenhouse systems leverage sensors to capture critical parameters such as temperature, humidity, soil moisture, and even electrical quantities that inform decisions for optimizing irrigation, fertilization, and pest management \cite{simo2022smartagricultureiotbased}. Although these systems provide promising enhancements in yield and resource optimization, they are often limited by issues of sensor placement accuracy, variable communication range requirements, and vulnerability to security breaches when interfacing directly with cloud platforms \cite{viswanatha2022implementationofiot, abunadi2022trafficawaresecuredcooperative}. These limitations highlight the need for robust data validation and advanced security mechanisms that can be addressed by integrating blockchain-based frameworks.
  \subsection{IoT Architectures Integrating AI, Blockchain, and Edge Computing}\label{subsec:lit-ai-bc-edge}
More recent studies have begun integrating AI with IoT systems to enhance the decision-making process in smart agriculture. For instance, several works employ deep learning and other machine learning techniques for crop disease detection, yield prediction, and nutrient management \cite{raju2022aselfpoweredrealtime, bakthavatchalam2022iotframeworkfor}. Furthermore, researchers have proposed domain-agnostic frameworks, such as the Monitoring and Control Framework (MCF), which leverage modular and open-source components to improve scalability and interoperability across different IoT domains, including agriculture \cite{senoo2023monitoringandcontrol, senoo2023monitoringandcontrol}. While these integrated systems effectively combine sensor-derived data with AI analytics, they still face limitations in ensuring real-time data security, resistance to cyber-attacks, and the energy constraints imposed by remote sensor nodes \cite{abunadi2022trafficawaresecuredcooperative, ouafiq2022datamanagementand}. More importantly, interoperability between heterogeneous devices remains a significant challenge. These drawbacks have motivated ongoing research into blockchain-enabled IoT frameworks that employ parallel transaction models and consensus mechanisms to secure data, improve trust among distributed nodes, and realize quality-of-service (QoS) guarantees \cite{quy2022iotenabledsmartagriculture, abunadi2022trafficawaresecuredcooperative}.


  \subsection{Energy Efficiency in Blockchain–IoT Agriculture}\label{subsec:lit-energy}
Recent studies have focused on reducing blockchain’s energy footprint while maintaining data security and transparency in agricultural IoT applications. Munaganuri et al. \cite{munaganuri2025designofan} propose an integrated model that combines Long Short-Term Memory (LSTM) networks, IoT sensors communicating via the low-power LoRaWAN protocol, and Hyperledger Fabric blockchain, complemented by reinforcement learning with Deep Q-Networks (DQN) to optimize irrigation scheduling. Their field trials reported a 20\% reduction in water usage accompanied by a 12\% increase in crop yield, demonstrating not only improved resource efficiency but also substantial energy savings through the adoption of low-power wireless communication protocols. Other studies have proposed lightweight blockchain architectures such as Easychain \cite{bapatla2023easychainaniotfriendly}, which is specifically designed for IoT environments with limited computational resources. These lightweight approaches focus on reducing the overhead associated with conventional consensus algorithms and employ alternative mechanisms (e.g., Proof-of-Stake or Proof-of-Authority) to further reduce energy consumption \cite{alkhateeb2022hybridblockchainplatforms, tahayur2024enhancingelectronicagriculture}.

However, challenges remain. For instance, while many researchers report favorable results in simulation environments or small-scale field trials, scalability concerns persist when these energy-efficient models are considered for large-scale or diverse agricultural landscapes \cite{akella2023asystematicreview, munaganuri2025designofan}. Furthermore, the integration of renewable energy sources and optimized resource allocation strategies presents additional opportunities to further reduce the energy footprint, yet require more rigorous experimental validation \cite{tahayur2024enhancingelectronicagriculture, mwewa2024blockchaintechnologya}.

\subsection{Usability and Farmer Adoption of Blockchain-Enabled Smart Farming Systems}
The usability of blockchain systems is critical for achieving broad adoption among smallholder farmers and other stakeholders in agriculture. Researchers have applied models such as the Technology Acceptance Model (TAM) and its extensions \cite{ninsiima2025determinantsofsmallholder, mwewa2024blockchaintechnologya} to assess the determinants of blockchain adoption. These studies consistently identify factors such as perceived usefulness, ease of use, and subjective norm as significant predictors of a farmer’s intention to use blockchain-based systems \cite{ninsiima2025determinantsofsmallholder, mwewa2024blockchaintechnologya}. For example, a study by Ninsiima et al. \cite{ninsiima2025determinantsofsmallholder} reports that blockchain technologies, when integrated with user-friendly interfaces and supported by training programs, can effectively reduce information asymmetry and improve trust between farmers and buyers.

User-centered design is therefore pivotal. An example is provided by Price-Torrejón et al. \cite{pricetorrejon2025designofa}, who developed a blockchain-enabled web application prototype designed to optimize traceability in agricultural supply chains. Their evaluation using the System Usability Scale (SUS) yielded an impressive average score of 90, confirming the prototype’s excellent usability and its potential for adoption among low-technical users. Yet despite these positive findings, several limitations continue to impede widespread adoption. High implementation costs, infrastructure deficiencies, and the complexity of blockchain interfaces remain significant barriers, especially in developing regions where digital literacy is limited \cite{ninsiima2025determinantsofsmallholder, akella2023asystematicreview}. Other studies suggest that blockchain-as-a-service (BaaS) platforms might alleviate these hurdles by offloading technical complexities onto third-party providers, though economic and regulatory challenges persist \cite{akella2023asystematicreview}.

  \subsection{Blockchain-Enabled Traceability and Supply Chain Integration}\label{subsec:lit-traceability}
Blockchain’s decentralized ledger technology has been extensively applied in agri-food supply chains to improve traceability and transparency. Numerous real-world applications—such as BeefLedger, Pagonis Dairy, FarMarket, and HARA—demonstrate blockchain’s ability to track agricultural products from pre-harvest through to post-harvest stages \cite{akella2023asystematicreview, mwewa2024blockchaintechnologya}. Such traceability systems leverage smart contracts and integrate data acquired from IoT sensors (RFID, GPS, NFC) to manage provenance information, ensure food safety, and simplify financial transactions among stakeholders \cite{ellahi2023blockchainbasedframeworksfor, akella2023asystematicreview}.

These systems provide significant benefits, including enhanced food quality, reduced fraud, and improved consumer confidence through transparent record-keeping. However, limitations are also evident. For instance, many existing systems are designed to address specific facets of the agricultural process rather than offering full-stack solutions that cover the entire supply chain \cite{akella2023asystematicreview}. Additionally, interoperability issues among disparate ICT systems, scalability challenges, and the lack of unified global standards have been highlighted as key obstacles for widespread adoption \cite{akella2023asystematicreview}. \begin{figure}[htbp]
  \centering
  \includegraphics[width=\linewidth]{figa.png}
  \caption{Schematic representation of a blockchain-based traceability framework, illustrating the integration of IoT sensors and smart contracts for end-to-end supply chain monitoring~\cite{sakthivel2024enhancingtransparencyand}.}
  \label{fig:figa}
\end{figure}

As shown in \autoref{fig:figa}, the schematic representation of a blockchain-based traceability framework illustrates the integration of IoT sensors and smart contracts for end-to-end supply chain monitoring.


\subsection*{Blockchain-enabled IoT Architectures in Smart Agriculture}  Several recent studies have proposed blockchain-enabled IoT frameworks tailored to the unique demands of agricultural applications. Ali and Sofi 2022 present a pioneering architecture for the saffron agri-value chain in which IoT nodes are organized hierarchically. Lightweight IoT devices is designated for data acquisition during cultivation, processing, and logistics, while edge gateways -- acting as full nodes -- manage local copies of the blockchain ledger, perform computationally expensive tasks, and coordinate consensus operations. Their design employs a selective consensus mechanism that dynamically chooses the most appropriate protocol based on network size, data volume, computational limits, latency, throughput, and network overhead \cite{ali2022blockchainenabledarchitecture}. In a subsequent study also by Ali and Sofi 2022, the blockchain-enabled IoT framework is augmented to support end-to-end traceability for the saffron supply chain, with a focus on government-certified registration of stakeholders and enhanced quality control. Both studies underscore that although permissioned blockchain architectures substantially reduce the risk of data tampering and manipulation, heavy reliance on edge gateways to perform full-node tasks may create potential bottlenecks or single points of failure, particularly during high data throughput events \cite{ali2022blockchainenabledarchitecture}.  Another critical aspect of blockchain-enabled smart agriculture is the integration of decentralized architectures with IoT sensor networks for real-time monitoring of environmental parameters. Recent proposals stress that while IoT devices are excellent for capturing data, their limited computational and storage capabilities require that heavy blockchain tasks be offloaded to robust infrastructure nodes. The selective consensus approach, which dynamically adapts based on current network conditions, allows for an optimal trade-off between security and resource utilization \cite{ali2022blockchainenabledarchitecture}. For instance, protocols such as Practical Byzantine Fault Tolerance (PBFT) and variants like Secure Data Trading Ecosystem (SDTE) or Proof-of-Honesty (PLEDGE) are preferred for smaller networks, whereas Delegated Proof-of-Stake (DPoS), Proof-of-Elapsed-Time (PoET), and Tendermint are more appropriate for larger networks. This adaptive selection leverages edge computing to consolidate transaction processing and ensures that IoT sensors maintain only minimal state information, such as block headers, thereby mitigating resource constraints \cite{ali2022blockchainenabledarchitecture}.  Furthermore, blockchain traceability models have been extended beyond standalone agricultural operations into the broader agri-food supply chain. In this context, blockchain-enabled IoT architectures support secure data sharing among heterogeneous stakeholders---from farmers and processors to distributors and consumers---by providing immutable records of provenance, quality control, and logistics \cite{demestichas2020blockchaininagriculture, bosona2023theroleof}. This integration not only boosts consumer trust through verifiable ``farm-to-fork'' transparency but also enables regulatory compliance and fraud prevention in the food industry.

\subsection*{ Consensus Mechanisms and Parallel Transaction Models}

Consensus mechanisms are the critical enablers that underpin blockchain's security and data integrity. Conventional algorithms such as Proof of Work (PoW) are unsuitable for resource-constrained IoT environments due to their high computational costs and energy consumption. Recent studies have turned to alternative consensus protocols that balance efficiency with robust security. Ali and Sofi 2022 propose the application of consensus methods such as PBFT for small-scale networks and DPoS or PoET for large-scale networks. Their system features a dynamic consensus selection algorithm that evaluates key performance metrics such as latency, throughput, and network overhead to determine the optimal protocol in real time \cite{ali2022blockchainenabledarchitecture}.

In a related development, innovative parallel transaction models based on the Chinese Remainder Theorem (CRT) have been introduced to improve the scalability of blockchain systems in agriculture. By partitioning transaction processing across multiple parallel channels, these CRT-based models increase transaction throughput and enhance overall QoS, facilitating real-time applications like precision crop monitoring and automated irrigation control \cite{ali2022blockchainenabledarchitecture}. Despite the promise shown by these models, empirical validation under actual field conditions remains limited. Researchers caution that the dynamic adjustment of consensus parameters and the management of parallel streams must be rigorously tested to ensure consistent performance across diverse network conditions \cite{ali2022blockchainenabledarchitecture}. This represents a major challenge as agricultural IoT networks are typically heterogeneous and subject to variable connectivity and environmental factors.

The literature further indicates that incorporating lightweight consensus mechanisms into blockchain-enabled IoT frameworks is pivotal in tackling scalability concerns. Recent reviews have highlighted that even with selective consensus approaches, the resource limitations of IoT devices necessitate offloading heavy computations to more capable edge nodes \cite{ali2022blockchainenabledarchitecture}. Moreover, the need for parallel processing architectures has motivated the integration of CRT-based partitioning strategies that can concurrently process transactions, thereby reducing delays and minimizing the risk of network congestion. However, such models are still in the prototype stage and require comprehensive evaluations regarding their responsiveness, fault tolerance, and security resilience in large-scale deployments \cite{ali2022blockchainenabledarchitecture}.

\subsection*{Blockchain-based Smart Farm Security Frameworks}

As smart agriculture increasingly relies on distributed IoT sensor networks, the security of these systems becomes critically important. Agricultural IoT devices are prime targets for cyber-attacks that could compromise not only the integrity of data but also the physical operations on farms. In response to these vulnerabilities, Aliyu and Liu 2023 have developed a blockchain-based smart farm security framework that integrates an Arduino sensor kit, cloud services such as AWS, and Ethereum-based smart contracts to detect and mitigate poisoning attacks in real time \cite{aliyu2023blockchainbasedsmartfarm}.

Their framework employs secure data encryption and real-time notifications to inform farmers of suspicious activities or sensor malfunctions. A notable innovation in this work is the use of an exchange blockchain---potentially a more powerful platform like Cardano---to lower processing delays further and enhance responsiveness. Simulation results indicate that higher accepted transaction rates correlate with faster alarm induction, essential for preventing cyber-attacks \cite{aliyu2023blockchainbasedsmartfarm}. However, the study is limited by a relatively small experimental dataset and challenges related to scaling node transaction processing. Furthermore, while the use of neural network-based classifiers for predicting and detecting attacks is suggested, additional validation is required to ensure robustness against emerging threats \cite{aliyu2023blockchainbasedsmartfarm}.

Other studies have echoed these concerns, emphasizing that blockchain's inherent immutability and decentralized verification provide formidable defenses against data tampering and unauthorized access. The integration of cryptographic primitives and secure smart contracts helps to ensure that any modifications to the sensor data are both detectable and irreversible \cite{aliyu2023blockchainbasedsmartfarm, demestichas2020blockchaininagriculture}. Yet, the reliance on simulation-based evaluations rather than extensive real-world deployment remains a critical gap in the literature. In summary, while blockchain-enabled security frameworks offer promising mechanisms to secure smart farm operations, scalability, experimental validation, and advanced attack detection remain areas for further research.

\subsection*{Blockchain Traceability in Agri-food Supply Chains}

One of the most mature and well-studied applications of blockchain technology in agriculture is food traceability. In an environment where food safety and authenticity are critical, blockchain's decentralized ledger provides an immutable record that supports end-to-end traceability of products from cultivation through processing, distribution, and final sale. Bosona and Gebresenbet 2023 review the role of blockchain in promoting traceability systems in agri-food production and supply chains, arguing that decentralized data management overcomes the biases and inherent vulnerabilities of traditional centralized systems \cite{bosona2023theroleof}. Their work emphasizes that integrating IoT technologies (e.g., RFID and sensors) with blockchain not only enhances transparency but also facilitates real-time monitoring of food safety parameters.

Similarly, Demestichas et al. 2020 analyze various blockchain consensus mechanisms and their applicability to agri-food traceability. Their review highlights that while public blockchains provide high transparency, the energy inefficiencies and scalability problems of traditional methods such as PoW necessitate the use of permissioned blockchain models that offer more efficient consensus protocols \cite{demestichas2020blockchaininagriculture}. Key limitations identified in these reviews include challenges associated with integrating blockchain solutions into legacy systems, ensuring consistent data standardization across multiple stakeholders, and achieving scalability in the presence of high transaction volumes \cite{demestichas2020blockchaininagriculture}.

Other reviews \cite{demestichas2020blockchaininagriculture, chandan2023achievingunsdgs} have further underscored that while blockchain-enabled traceability systems are theoretically robust, practical implementation is often complicated by heterogeneous data sources, lack of comprehensive interoperability frameworks, and difficulties in meeting diverse regulatory requirements. Overall, blockchain-based traceability in the agri-food supply chain system is recognized as a transformative technology that addresses food fraud, ensures quality control, and fosters consumer trust; however, these systems are still impeded by integration and scalability challenges that call for further investigation and optimization.

\subsection*{Emerging Approaches: Metaheuristics and Advanced Consensus}

Beyond the adaptive consensus mechanisms and parallel transaction models currently proposed, several recent studies have integrated advanced computational techniques to further enhance blockchain performance in smart agriculture. Khan et al. 2022 have proposed a distributed architecture that combines metaheuristic algorithms---specifically genetic algorithms for process scheduling---with blockchain technology implemented on a Hyperledger Sawtooth private network \cite{khan2022ablockchainand}. Their framework optimizes process scheduling and commodity forecasting through machine learning regression methods while automating stakeholder registration and ledger updates via smart contracts. Although the integration of metaheuristic techniques results in improved process efficiency and ledger preservation, limitations remain, including interoperability challenges among distributed nodes, limited node registration capacity, and incomplete encryption mechanisms \cite{khan2022ablockchainand}.

Other approaches have focused on incorporating deep learning techniques with blockchain technology to improve real-time transaction validation and streamline process optimization. Hybrid models that couple blockchain with advanced regression or neural network methods have shown promise in terms of forecasting accuracy and decision support; however, they are still constrained by scalability issues in node transaction processing and the complexity of interfacing heterogeneous IoT devices with these advanced algorithms \cite{khan2022ablockchainand, rana2020blockchainbasedtraceabilityand}. These investigations underscore that while metaheuristics and deep learning methods offer enhanced performance for agricultural data processing, the deployment of such complex systems in field conditions requires further experimental validation.

A common challenge in these emerging architectures is the integration of heterogeneous data sources from multiple IoT devices and ensuring that the blockchain's decentralized architecture can process and verify these data streams in real time. While these advanced techniques have been successfully demonstrated in controlled environments, scalability, interoperability, and long-term robustness remain open research questions \cite{khan2022ablockchainand}.

\subsection*{Crop Monitoring and IoT-driven Agricultural Efficiency}

Crop monitoring represents one of the most critical applications of IoT in agriculture and is greatly enhanced by blockchain's secure recordkeeping. Lin et al. 2018 have demonstrated a blockchain-based system that integrates IoT sensors to capture environmental data---including temperature, humidity, soil pH, and GPS coordinates---to support precision agriculture practices such as smart irrigation, fertilization, and pest control \cite{demestichas2020blockchaininagriculture, ali2022blockchainenabledarchitecture}. The immutable ledger provided by blockchain ensures that sensor data remain tamper-proof, enabling robust analysis of crop health and yield predictions.

Nevertheless, these systems face inherent limitations due to the on-chain storage of high-volume sensor data, which can strain the network in terms of processing speed and storage capacity. The heterogeneity among IoT devices means that consensus protocols may perform unevenly, leading to delays in data processing and reduced real-time responsiveness. As a solution, CRT-based parallel transaction models have been proposed that partition transaction processing into parallel streams, thereby alleviating throughput bottlenecks and enhancing quality-of-service (QoS) for real-time crop monitoring \cite{ali2022blockchainenabledarchitecture}. Although the idea of partitioning and concurrently processing data is conceptually promising, its real-world effectiveness under varying agricultural conditions still needs rigorous empirical evaluation. Issues such as dynamic traffic loads, network instability, and heterogeneous node performance continue to be significant challenges for such models \cite{ali2022blockchainenabledarchitecture}.

These limitations highlight a broader trend in blockchain-enabled crop monitoring solutions: while the integration of blockchain with IoT clearly enhances security and traceability, the scalability of such architectures remains a critical bottleneck that must be resolved before widespread adoption in precision agriculture.

\subsection*{Comparative Analysis and State-of-the-Art Comparison Table}

In order to provide clarity over the diverse approaches and methodologies reported in current literature, the tables below summarize a representative sample of recent studies.

\begin{table}[H]
\caption{Comparison of Selected Papers in Blockchain-enabled Smart Agriculture}
\label{tab:comparison}
\centering
\fitToPage{
\begingroup
\setlength{\tabcolsep}{4pt}
\renewcommand{\arraystretch}{1.15}
\footnotesize
\begin{tabularx}{\textwidth}{@{}P{0.18\textwidth}YYY@{}}
\toprule
\textbf{Paper} & \textbf{Key Findings} & \textbf{Methodologies} & \textbf{Limitations} \\
\midrule
Ali and Sofi (2022) & Proposes dynamic consensus selection based on network scale to offload heavy computations to full nodes, enhancing scalability & Adaptive consensus protocols for IoT-based saffron agri-value chain; integration with edge gateways & Edge gateways may become bottlenecks and single points of failure under high throughput \\
\addlinespace
Ali and Sofi (2022) & Integrates lightweight IoT data acquisition with permissioned blockchain ledger management for improved traceability & Lightweight IoT sensors with permissioned blockchain (e.g., Hyperledger) managed through edge gateways & Scalability concerns under high data volume; dependence on centralized ledger management impairs resilience \\
\addlinespace
Aliyu and Liu (2023) & Leverages secure smart contracts and real-time alerts to detect poisoning attacks and protect sensor networks & Ethereum smart contracts for automation; neural network-based classification for threat detection & Limited experimental data; challenges in scaling node transaction processing; validation of ML models not comprehensive \\
\addlinespace
Bosona and Gebresenbet (2023) & Demonstrates that decentralized traceability enhances food safety and transparency in agri-food chains & Review of IoT and RFID integration with blockchain for end-to-end traceability & Inconsistent data input from diverse sources; standardization challenges due to node heterogeneity \\
\addlinespace
Khan et al. (2022) & Combines genetic algorithms with a blockchain system to optimize process scheduling and commodity forecasting & Hyperledger Sawtooth-based system with genetic algorithms and smart contracts for forecasting & Interoperability issues among nodes; limited node registration; incomplete encryption strategies \\
\addlinespace
Demestichas et al. (2020) & Employs smart contracts and IoT sensor integration to secure end-to-end food traceability and improve compliance & Use of smart contracts (e.g., Ethereum) for data logging; IoT sensors for environmental monitoring & Low on-chain throughput; variable sensor performance due to heterogeneous IoT networks impacting efficiency \\
\bottomrule
\end{tabularx}
\endgroup
}
\end{table}

\begin{table}[H]
\caption{Summary of IoT-based Precision Agriculture and Sensor Networks}
\label{tab:summary-iot-precision}
\centering
\fitToPage{
\begingroup
\setlength{\tabcolsep}{4pt}
\renewcommand{\arraystretch}{1.15}
\footnotesize
\begin{tabularx}{\textwidth}{@{}P{0.18\textwidth}YYY@{}}
\toprule
\textbf{Paper} & \textbf{Key Findings} & \textbf{Methodologies} & \textbf{Limitations} \\
\midrule
Akhter et al. (2022) & Enhanced crop disease detection and irrigation scheduling via sensor networks and ML pipelines & Deployment of WSNs (e.g., soil moisture sensors) combined with machine learning models (e.g., CNN) & Reliance on simulated data; limited field validation; UI ergonomics underexplored \\
\addlinespace
Atalla et al. (2023) & Evaluates WSN performance for both stationary (olive) and mobile (livestock) agricultural contexts & Performance analysis of Wireless Sensor Networks in fixed and mobile farm scenarios; KPIs like latency and throughput & Simulation-centric validation; sparse real-world throughput/latency baselines \\
\addlinespace
Bayih et al. (2022) & Developed a low-cost sensing system with emphasis on precision irrigation and farmer accessibility & Design and deployment of low-cost IoT sensor nodes for soil and environmental monitoring & Threshold-based logic leads to energy inefficiencies; high calibration/maintenance overheads \\
\addlinespace
Simo et al. (2022) & Created a low-cost device for automated greenhouse climate monitoring and control & Design of a custom IoT device with sensors for temperature, humidity, and control actuators for greenhouses & Limited data standardization; range/security concerns on the radio link (e.g., LoRa) \\
\addlinespace
Raju et al. (2022) & Implemented a cloud-integrated, energy-aware monitoring system with self-powering capability & Use of NRF24L01 radio modules; integration of energy harvesting (solar); cloud data dashboard & ~200m line-of-sight range limit; scalability and sustained energy budget not proven at scale \\
\bottomrule
\end{tabularx}
\endgroup
}
\end{table}

\begin{table}[H]
\caption{Summary of Integrated IoT Architectures and AI–Blockchain Frameworks}
\label{tab:summary-iot-ai-bc}
\centering
\fitToPage{
\begingroup
\setlength{\tabcolsep}{4pt}
\renewcommand{\arraystretch}{1.15}
\footnotesize
\begin{tabularx}{\textwidth}{@{}P{0.18\textwidth}YYY@{}}
\toprule
\textbf{Paper} & \textbf{Key Findings} & \textbf{Methodologies} & \textbf{Limitations} \\
\midrule
Quy et al. (2022) & Proposes a layered architecture integrating sensing, cloud, and AI analytics for end-to-end data flow & Design of a multi-tier IoT architecture (sensing, network, cloud, application) with AI components & Mostly simulation-validated; interoperability across vendors under-specified \\
\addlinespace
Senoo et al. (2023) & Developed an open-source, modular stack targeting scalable, interoperable IoT deployments & Creation of a domain-agnostic, modular software and hardware framework for IoT monitoring and control & Early adoption stage; challenges with device heterogeneity and standards convergence remain open \\
\addlinespace
Bakthavatchalam et al. (2022) & Pushes embedded AI on low-power devices for long-duration monitoring and on-device decisions & Implementation of TinyML models on microcontroller-based IoT devices for local inference & Constrained by device energy budget; data-quality variance; high integration complexity at the edge \\
\addlinespace
Quy et al. (2022) & Highlights blockchain for data integrity, decentralization, and transaction auditability in agri-IoT & Analytical review of blockchain technology (e.g., Ethereum, Hyperledger) applied to IoT agriculture scenarios & Limited practical implementation details; lacks full-scale field trials and performance metrics \\
\addlinespace
Rahaman et al. (2024) & Integrates privacy-aware AI with cryptographic controls to secure the entire data life-cycle & Combination of differential privacy, federated learning, and lightweight cryptography for IoT data & Potential overhead/scalability issues in diverse field settings; standardization gaps exist \\
\bottomrule
\end{tabularx}
\endgroup
}
\end{table}

\begin{table}[H]
\caption{Summary of Energy Efficiency in Blockchain–IoT Agriculture}
\label{table:energy}
\centering
\fitToPage{
\begingroup
\setlength{\tabcolsep}{4pt}
\renewcommand{\arraystretch}{1.15}
\footnotesize
\begin{tabularx}{\textwidth}{@{}P{0.18\textwidth}YYY@{}}
\toprule
\textbf{Paper} & \textbf{Key Findings} & \textbf{Methodologies} & \textbf{Limitations} \\
\midrule
Munaganuri et al. (2025) & Achieved 20\% reduction in water usage and 12\% yield increase via energy-efficient LoRaWAN and blockchain & Integration of graph-based models, LSTM networks, LoRaWAN communication, and blockchain coordination & Faces scalability and heterogeneous integration challenges in real-field deployments \\
\addlinespace
Bapatla et al. (2023) & Designed a lightweight blockchain that reduces device-side energy and compute overhead for authentication & Development of "EasyChain", a custom blockchain protocol with an energy-efficient consensus mechanism & Interoperability with legacy systems and integration issues at scale are significant concerns \\
\addlinespace
Bodkhe et al. (2022) & Highlights energy constraints of irrigation workloads; motivates optimized consensus and data batching & Analysis of precision irrigation workflows and proposal of blockchain consensus optimizations for agri-IoT & Lacks large-scale, real-field validation to substantiate the proposed theoretical models \\
\addlinespace
Mustafa et al. (2024) & Multi-layer optimization (routing, crypto, analytics) conserves device energy in IoT networks & A cross-layer framework combining secure routing protocols, lightweight cryptography, and efficient data analytics & Simulation-heavy validation; robustness and scalability in heterogeneous farms remain to be proven \\
\bottomrule
\end{tabularx}
\endgroup
}
\end{table}

\begin{table}[H]
\caption{Summary of Usability and Farmer Adoption Studies}
\label{table:usability}
\centering
\fitToPage{
\begingroup
\setlength{\tabcolsep}{4pt}
\renewcommand{\arraystretch}{1.15}
\footnotesize
\begin{tabularx}{\textwidth}{@{}P{0.18\textwidth}YYY@{}}
\toprule
\textbf{Paper} & \textbf{Key Findings} & \textbf{Methodologies} & \textbf{Limitations} \\
\midrule
Ninsiima et al. (2025) & Technology Acceptance Model (TAM) factors significantly shape farmers' intentions to adopt blockchain & Survey-based study using structural equation modeling (SEM) to analyze adoption determinants among barley farmers & Focuses on intention (not post-adoption behavior); region-specific sample limits generalizability \\
\addlinespace
Akella et al. (2023) & Identifies collaboration/trust as enablers; cost, regulation, complexity as key adoption barriers & A systematic literature review (SLR) synthesizing barriers and enablers for blockchain adoption & Broad thematic synthesis; lacks granular, farmer-level usability and interaction data \\
\addlinespace
Price-Torrejon et al. (2025) & Developed a blockchain-based web app with high usability (SUS~$\approx$~90) for low-tech users & Design and development of a web application with a focus on UI/UX for traceability; evaluated via System Usability Scale & Tested with manual data entry; external validity to live, large-scale operations is limited \\
\addlinespace
Mwewa et al. (2024) & Transparency from blockchain can reduce costs and build trust along agri supply chains & A review study analyzing the impact of blockchain on efficiency and transparency in agricultural supply chains & Digital literacy and upfront costs identified as major adoption hurdles, especially for SMEs and small farms \\
\bottomrule
\end{tabularx}
\endgroup
}
\end{table}

\begin{table}[H]
\caption{Summary of Consensus Mechanisms for IoT Systems}
\label{tab:summary_consensus}
\centering
\fitToPage{
\begingroup
\setlength{\tabcolsep}{4pt}
\renewcommand{\arraystretch}{1.15}
\footnotesize
\begin{tabularx}{\textwidth}{@{}P{0.18\textwidth}YYY@{}}
\toprule
\textbf{Paper} & \textbf{Key Findings} & \textbf{Methodologies} & \textbf{Limitations} \\
\midrule
Ali and Sofi (2022) & Proposes selective consensus mechanisms to balance security and resource constraints in IoT & A three-tier BIoT architecture that dynamically selects consensus protocols (e.g., PoS, PoA) based on network scale & Performance limited by heterogeneous IoT device capabilities; challenges in standardizing consensus selection \\
\addlinespace
Arifeen et al. (2022) & Explores ML-enhanced consensus to reduce energy consumption and improve transaction validation & Use of autoencoder neural networks to improve the efficiency and security of consensus mechanisms & Complexity in verification processes; increased computational overhead from ML integration \\
\addlinespace
Guo et al. (2022) & Proposes a protocol combining hierarchical grouping and location-awareness to reduce latency & Design of LH-Raft protocol, using hierarchical group formation and location metrics for candidate selection & Requires further validation in large-scale deployments; trade-offs between local and global consensus parameters \\
\addlinespace
Bryant (2022) & Reviews lightweight consensus protocols that avoid energy-intensive PoW, emphasizing fairness & Comparative analysis of alternative consensus mechanisms like PoET and PBFT variants for IoT security & Dependency on proprietary hardware (e.g., Intel's SGX for PoET) limits broader applicability \\
\addlinespace
Khan et al. (2022) & Develops an ontology to categorize and compare IoT-friendly consensus algorithms & Creation of the CONIoT ontology, a formal taxonomy for classifying blockchain consensus algorithms for IoT & Complexity in mapping dynamic consensus variations; partial adherence to best practices in ontology design \\
\bottomrule
\end{tabularx}
\endgroup
}
\end{table}
\clearpage

\subsection*{QoS Enhancements in Blockchain-IoT Systems}
Several recent studies have proposed innovative blockchain models that integrate machine learning, bio-inspired algorithms and sidechain management to optimize QoS in IoT networks. For instance, Agrawal and Kumar \cite{agrawal2022mlsmbqsdesignof} introduce the MLSMBQS model---a machine learning--based split and merge blockchain model---designed specifically for securing IoT deployments while achieving improvements in throughput (up by up to 8.5\%--26.3\%), delay reduction (up to 15.3\%--24.8\% reduction), and lower energy consumption relative to traditional blockchain frameworks. Their approach exploits an Elephant Herding Optimization (EHO) algorithm and dummy traffic generation to differentiate and manage malicious versus regular network transactions. Although promising, the model has limitations concerning scalability; the evaluations have been largely confined to simulation environments with up to 500 nodes, and assumptions regarding standardized network conditions (e.g., two-ray ground propagation) may not hold in heterogeneous real-world deployments \cite{agrawal2022mlsmbqsdesignof}.

Other researchers, such as Gupta and Lakhwani \cite{gupta2025enhancingblockchainqualityofservice}, focus on smart contract frameworks and their impact on QoS in blockchain systems. They propose streamlined Solidity-based smart contracts that enhance transaction processing efficiency by incorporating features like reflection tokenomics, dividend token mechanisms and dynamic liquidity provisioning. While these studies report significant throughput improvements and energy savings, the limitations include scalability concerns when transitioning to large-scale IoT networks, as well as interoperability issues across heterogeneous blockchain protocols \cite{gupta2025enhancingblockchainqualityofservice}. Such challenges are accentuated when these models face high computational loads or require rapid transaction finality.

\subsection*{Lightweight Consensus Mechanisms for IoT}
In the context of resource-constrained IoT devices, lightweight consensus algorithms are essential for maintaining low latency and high throughput. Several studies have demonstrated that blockchain models using Delegated Proof of Stake (DPoS) can dramatically outperform traditional consensus algorithms such as Proof of Work (PoW) and Proof of Stake (PoS). For example, research by Ul Haque et al. \cite{haque2024ascalableblockchain} shows that DPoS achieves significantly higher transactions per second (TPS) and lower latency---even when the network scales to several thousand nodes. Their experimental evaluations indicate that DPoS can deliver TPS figures exceeding 500 for moderate networks with linear scalability. Nonetheless, these studies recognize inherent limitations, such as challenges in maintaining decentralization with a reduced validator set, possible vulnerabilities if delegate selection is compromised, and the inability to dynamically adapt to rapidly changing IoT network conditions \cite{haque2024ascalableblockchain}.

\subsection*{Queuing Theory Models and Network Scheduling}
A complementary approach to improving QoS in blockchain-IoT systems has been the application of queuing theory and scheduling algorithms to manage bandwidth allocation and traffic prioritization. Fawzy Habeeb et al. \cite{habeeb2022dynamicbandwidthslicing} propose a multi-level queuing model that differentiates between latency-sensitive and delay-tolerant applications, dynamically slicing network bandwidth to achieve up to 9$\times$ reduction in network latency. Such methodologies leverage queueing theory techniques to optimize resource utilization; however, they often simplify the network by assuming fixed traffic patterns or homogeneous node behavior, which might not translate to complex, heterogeneous IoT deployments \cite{habeeb2022dynamicbandwidthslicing}. Similar studies employ Generalized Processor Sharing (GPS) and advanced machine learning models to predict and manage network load \cite{zhang2024ondemandcentralizedresource}. Although these models are valuable for early-stage architectural design, they are commonly limited by their reliance on assumptions that fail to capture real-world complexities, such as variable network congestion and dynamic, unpredictable IoT traffic patterns.

\subsection*{Experimental Performance Analysis of Hyperledger Fabric}
A number of investigations have concentrated on evaluating the performance of permissioned blockchain platforms, notably Hyperledger Fabric (HLF), for IoT applications. Experimental works \cite{pajooh2022experimentalperformanceanalysis} have detailed the impact of various configuration parameters---such as block size, batch-timeout, and endorsement policy---on critical QoS metrics including throughput and latency. For instance, studies by Honar Pajooh et al. \cite{pajooh2022experimentalperformanceanalysis} report that larger block sizes can increase throughput by validating multiple transactions simultaneously, but may also introduce latency if the system waits too long to form a block. Their analyses underscore that proper tuning of parameters is essential for achieving an optimum balance between speed and security. A key limitation identified in these studies is that experimental setups are often based on simulated or controlled environments (e.g., using AWS testbeds) and that observed performance may degrade in real-world IoT deployments with highly variable workloads \cite{pajooh2022experimentalperformanceanalysis}.

\subsection*{Hybrid Architectural Approaches}
Other researchers have explored hybrid architectures that integrate on-chain and off-chain data management to improve scalability and overall performance. For example, studies by Ul Haque et al. \cite{haque2024ascalableblockchain} propose frameworks in which a local, lightweight blockchain is maintained at the IoT gateway, while a public blockchain stores hash references for data stored off-chain via systems like IPFS. This architecture reduces on-chain data load and minimizes latency within the IoT ecosystem. However, limitations remain in ensuring real-time auditing and transparent access to data when significant information is stored off-chain. Furthermore, managing the interoperability between on-chain and off-chain components introduces additional complexity, especially when scaling to thousands of IoT devices \cite{haque2024ascalableblockchain}.

\subsection*{SDN-IoT Integration and Heterogeneity Management}
The use of Software Defined Networking (SDN) in IoT environments has been applied to address network heterogeneity and QoS constraints. Recent work by Zafar et al. \cite{zafar2023anadvancedstrategy} addresses the challenges posed by heterogeneous controllers in SDN-IoT networks. Their approach groups distributed SDN controllers according to similar service rates using queuing models such as the M/M/1 model to better predict controller response times and ensure more reliable flow scheduling. While innovative, these solutions suffer from limitations such as their dependence on idealized network conditions and limited validation in multi-controller scenarios under real-world traffic fluctuations. Such heterogeneity challenges directly impact QoS---manifesting as increased latency and reduced throughput---thus calling for more robust, dynamically adaptable control mechanisms \cite{zafar2023anadvancedstrategy}.

\subsection*{Similarities and Divergences Among the Reviewed Approaches}
An analysis of the reviewed literature demonstrates several recurring themes. Most models, whether focused on blockchain structure (e.g., split \& merge sidechains) or on network scheduling via queuing theory, emphasize the need to balance security, energy efficiency, and real-time performance. Several studies employ machine learning methods to dynamically adjust QoS parameters in response to network conditions \cite{agrawal2022mlsmbqsdesignof,zhang2024ondemandcentralizedresource}. At the same time, lightweight consensus mechanisms and architectural decisions---such as off-chain storage and hybrid blockchains---are common strategies to mitigate computational overhead and latency \cite{haque2024ascalableblockchain}. However, most papers note limitations related to scalability (with many evaluations restricted to relatively small networks), reliance on simulation environments that may not capture real-world heterogeneity, and the challenges inherent in mapping theoretical queuing models to dynamic, unpredictable IoT traffic \cite{habeeb2022dynamicbandwidthslicing,zhang2024ondemandcentralizedresource}. These common issues illustrate the critical need for robust solutions when transitioning from experimental to large-scale deployments.

\section{Critical Analysis and Gap Mapping}\label{sec:critical-gap}
The state-of-the-art review reveals that recent research in blockchain-enabled IoT frameworks for smart agriculture has achieved notable advancements in secure supply chain traceability, remote crop monitoring, and lightweight consensus mechanisms for resource-constrained environments. However, several critical gaps remain that present opportunities for Hyperledger-based solutions.

One major limitation identified across many studies is scalability. Research papers \cite{aliyu2023blockchainbasedsmartfarm, ellahi2023blockchainbasedframeworksfor, sakthivel2024enhancingtransparencyand} note that while blockchain effectively ensures data integrity and traceability in controlled environments, real-world deployments in large-scale, heterogeneous IoT networks remain challenging. Hyperledger Fabric, with its permissioned and modular architecture, can mitigate scalability issues by reducing computational overhead and allowing for channel-based private transactions. Moreover, Hyperledger's pluggable consensus mechanisms (e.g., Raft) can be tuned to optimize network latency and throughput, offering a potential solution to the energy constraints identified in many studies \cite{aliyu2023blockchainbasedsmartfarm, ellahi2023blockchainbasedframeworksfor, sakthivel2024enhancingtransparencyand}.

Another gap is in consensus mechanism efficiency. Many works \cite{ali2022blockchainenabledarchitecture, morais2023surveyonintegration, khan2022ablockchainand} propose lightweight consensus algorithms; however, these proposals often lack extensive real-world validation. Hyperledger Fabric's consensus model, which utilizes crash fault-tolerant protocols like Raft or PBFT alternatives, provides an already mature and scalable approach that can be further optimized for IoT applications. In addition, Hyperledger's support for chaincode upgrades allows for iterative improvements in consensus design without interrupting network operations, a feature that can help overcome the limitations noted in simulation-based studies \cite{saha2022blockchainchangingthe, ali2022blockchainenabledarchitecture}.

Quality of service (QoS) is another area where many studies \cite{ellahi2023blockchainbasedframeworksfor, ellahi2023blockchainbasedframeworksfor, khan2022ablockchainand} observe high latency and throughput issues, particularly due to off-chain and on-chain data processing delays. Hyperledger frameworks can address these challenges by offloading computationally intensive tasks to edge and fog computing layers while maintaining a secure, immutable ledger for critical transactions. Furthermore, Hyperledger Fabric supports private data collections and channels that can enhance data privacy and reduce the volume of data processed on the main ledger. This can lead to lower latency and more efficient queuing of transactions, thereby improving QoS \cite{khan2022ablockchainand, sizan2505asecuredtriad}.

Lastly, interoperability and heterogeneity remain prominent limitations. The reviewed literature shows that integrating diverse IoT devices and legacy systems into a unified blockchain solution leads to complexity \cite{aliyu2023blockchainbasedsmartfarm, tang2024assessingblockchainand}. Hyperledger's modular design and established standards, such as Fabric's Membership Service Provider (MSP), provide robust mechanisms for authenticating and integrating different devices and platforms securely. This interoperability can allow wide-scale deployment across diverse agricultural environments and even across country borders where regulatory compliance is critical \cite{sakthivel2024enhancingtransparencyand, bosona2023theroleof}.

  \subsection{Hyperledger Mapping: Identifying Limitations and Potential Correspondences}\label{subsec:hyperledger-mapping}
Drawing from the limitations highlighted in the literature, the following key issues have been consistently recognized:

\begin{itemize}
\item \textbf{Scalability and Network Load:} Many studies \cite{atalla2023iotenabledprecisionagriculture, bayih2022utilizationofinternet, bayih2022utilizationofinternet} report that IoT systems deployed in agricultural settings struggle with scaling sensor networks over vast areas due to energy inefficiency, limited communication range, and data overload in cloud platforms.  
 – \textit{Mapping:} These scalability issues can be addressed by leveraging Hyperledger’s modular architecture and parallel transaction processing, which distribute the network load over multiple nodes while ensuring low-latency consensus.

\item \textbf{Interoperability and Heterogeneity:} Numerous works \cite{bayih2022utilizationofinternet, bayih2022utilizationofinternet, bayih2022utilizationofinternet} emphasize that the integration of heterogeneous devices—from off-the-shelf microcontrollers to specialized sensors—often suffers due to fragmented standards and communication protocols.  
 – \textit{Mapping:} Hyperledger’s permissioned blockchain can create common communication interfaces and smart contracts that enforce uniform data formats and interoperability across diverse devices.

\item \textbf{Energy Efficiency and Sensor Lifetime:} Studies on self-powered IoT systems \cite{raju2022aselfpoweredrealtime, bayih2022utilizationofinternet} highlight energy constraints, particularly in remote or smallholder settings where power availability is limited.  
 – \textit{Mapping:} Hyperledger-driven frameworks allow for decentralized energy management schemes, potentially integrating low-power consensus algorithms and incentivizing energy harvesting through secure token-based mechanisms without altering the core mapping process detailed here.

\item \textbf{Data Security and Privacy:} A critical vulnerability in smart agriculture IoT systems is the risk of data tampering, privacy breaches, and cyber-attacks—issues raised in several works \cite{abunadi2022trafficawaresecuredcooperative, rahaman2024privacycentricaiand, ouafiq2022datamanagementand}.  
 – \textit{Mapping:} Hyperledger Fabric and related Hyperledger projects offer robust cryptographic primitives, role-based access control, and secure consensus protocols that can mitigate these security risks.

\item \textbf{Simulation versus Real-world Validation:} Many architectures remain untested beyond simulated environments \cite{bayih2022utilizationofinternet, atalla2023iotenabledprecisionagriculture} and therefore fail to capture the complex dynamics of actual agricultural fields.  
 – \textit{Mapping:} Integration with Hyperledger can enhance real-world data integrity and auditability through aggregated transaction histories, ensuring that the system performance is verifiable in practical deployments.

\item \textbf{Data Integration and Quality:} The quality of sensor data and its subsequent integration into big data workflows is another recurring challenge \cite{ouafiq2022datamanagementand, ouafiq2022datamanagementand}.  
 – \textit{Mapping:} Hyperledger’s immutable ledger can serve as a trusted data source for post-processing, enabling better quality assurance and traceability of sensor data through consensus-driven verification.
\end{itemize}

These mapped correlations indicate that while traditional IoT architectures in agriculture encounter limitations in scalability, interoperability, energy efficiency, security, and data quality, the integration of a Hyperledger-based blockchain layer has the potential to alleviate many of these concerns by providing a secure, distributed, and standardized transactional framework.
  \subsection{Mapping Consensus Limitations to Hyperledger Solutions}\label{subsec:consensus-to-hl}
A recurring theme in the reviewed literature is that many consensus mechanisms developed for IoT suffer from excessive computational and energy overhead, limited scalability when deployed in heterogeneous networks, and challenges with latency and real-time responsiveness \cite{ali2022blockchainenabledarchitecture, bryant2022keychallengesin, guo2022ahierarchicaland}. In mapping these limitations to the solutions provided by Hyperledger, the following points are observed:
\begin{itemize}
    \item Scalability and efficiency concerns are addressed by Hyperledger Fabric's modular architecture, which supports pluggable consensus protocols (such as PBFT, Raft, and Kafka-based ordering) designed to function in permissioned networks with optimized throughput \cite{khan2022asurveyand, bryant2022keychallengesin}.
    \item Latency issues encountered in IoT deployments are mitigated by Hyperledger's support for parallel transaction validation and endorsement policies that can be tuned to balance speed with fault tolerance \cite{morais2023surveyonintegration, ali2022blockchainenabledarchitecture}.
    \item The heavy computational demands presented by conventional PoW or hybrid consensus approaches, as noted in multiple studies \cite{ali2022blockchainenabledarchitecture, bryant2022keychallengesin}, are partially overcome by employing permissioned blockchain models like Hyperledger Fabric that utilize more energy-efficient consensus models that do not depend on resource-intensive mining.
    \item Security limitations, particularly in defending against Sybil attacks and ensuring fault tolerance in dynamic IoT networks, are addressed by Hyperledger through rigorous membership services and identity management frameworks that provide robust access control and secure communication channels \cite{guru2023asurveyon, platt2023sybilinthe}.
    \item In addition, Hyperledger's design for enterprise applications-which emphasizes privacy, confidentiality, and efficient data sharing-is highly relevant to IoT-based smart agriculture, where sensitive data and decentralized decision making must be balanced with performance \cite{khan2022asurveyand, haque2024ascalableblockchain}.
\end{itemize}

These mappings indicate that while the surveyed consensus mechanisms have identified key limitations such as resource constraints and latency, Hyperledger's architecture offers targeted solutions, particularly in permissioned contexts, that can be adapted for IoT frameworks such as smart agriculture.
\begin{table}[H]
\caption{State-of-the-Art Comparison and Hyperledger Mapping}
\label{table5}
\centering
\fitToPage{
\begingroup
\setlength{\tabcolsep}{4pt}
\renewcommand{\arraystretch}{1.15}
\footnotesize
\begin{tabularx}{\textwidth}{@{}P{0.18\textwidth}YYY@{}}
\toprule
\textbf{Paper} & \textbf{Key Findings} & \textbf{Methodologies} & \textbf{Limitations} \\
\midrule
Various Authors (2022-2024) & Many frameworks demonstrate traceability and secure data sharing at small-scale but face scalability issues & Analysis of multiple blockchain frameworks for IoT applications & High energy consumption, limited throughput, and high latency in large-scale deployments \\
\addlinespace
Various Authors (2022-2024) & Lightweight consensus alternatives show promise in reducing resource overhead & Evaluation of consensus mechanisms like PoS, PoA, and custom lightweight protocols & Many proposals remain simulation-based with limited real-world testing; energy issues persist \\
\addlinespace
Various Authors (2022-2024) & Immutable ledger provides end-to-end traceability and provenance in supply chains & Implementation of blockchain for supply chain transparency and data integrity & Integration with heterogeneous legacy systems remains complicated; high latency in public networks \\
\addlinespace
Various Authors (2022-2024) & Off-chain storage and smart contract optimizations reduce latency and improve throughput & Design of hybrid on-chain/off-chain architectures with optimized smart contracts & Latency issues and network congestion during high data loads; limited QoS parameter optimization \\
\addlinespace
Various Authors (2022-2024) & Diverse IoT sensors and device networks can be integrated to create hybrid smart agriculture systems & Development of interoperability frameworks for multi-protocol IoT environments & Integration complexity and non-standardized protocols lead to fragmented implementations \\
\bottomrule
\end{tabularx}
\endgroup
}
\end{table}

\begin{table}[H]
\caption{State-of-the-Art Comparison of QoS in Blockchain-IoT Systems}
\label{table6}
\centering
\fitToPage{
\begingroup
\setlength{\tabcolsep}{4pt}
\renewcommand{\arraystretch}{1.15}
\footnotesize
\begin{tabularx}{\textwidth}{@{}P{0.18\textwidth}YYY@{}}
\toprule
\textbf{Paper} & \textbf{Key Findings} & \textbf{Methodologies} & \textbf{Limitations} \\
\midrule
Agrawal and Kumar (2022) & Enhances security and QoS in IoT with 8.5\%--26.3\% throughput improvements and lower latency & Machine learning-based split and merge blockchain model with Elephant Herding Optimization algorithm & Scalability evaluated only up to 500 nodes; reliance on simulation setups \\
\addlinespace
Gupta and Lakhwani (2025) & Integrates ERC20 enhancements and dividend token models for secure transactions & Streamlined Solidity-based smart contracts with reflection tokenomics and dynamic liquidity provisioning & Limited scalability and interoperability challenges across heterogeneous blockchain protocols \\
\addlinespace
Various Authors (2022-2024) & DPoS significantly outperforms PoS and PoW in IoT settings, achieving high TPS & Implementation of Delegated Proof of Stake consensus for IoT networks & Reduced decentralization; limited evaluation on heterogeneous devices \\
\addlinespace
Various Authors (2022-2024) & Achieves up to 9$\times$ reduction in network latency by dynamically allocating bandwidth & Dynamic bandwidth slicing techniques for time-critical IoT data streams & Models assume fixed traffic patterns; limited by simulated network dynamics \\
\addlinespace
Various Authors (2022-2024) & Block size, batch-timeout, and endorsement policy significantly impact throughput and latency & Experimental performance analysis of Hyperledger Fabric configurations & Results based on controlled environments; real-world IoT constraints not fully captured \\
\addlinespace
Various Authors (2022-2024) & Groups heterogeneous SDN controllers to reduce response time and improve flow scheduling & Software-Defined Networking approach with queuing theory for controller scheduling & Limited testing in multi-domain settings; dependent on ideal network conditions \\
\bottomrule
\end{tabularx}
\endgroup
}
\end{table}

This comparison underscores that while current literature successfully demonstrates many promising aspects of blockchain-enabled IoT systems, persistent challenges such as scalability, consensus efficiency, QoS, and interoperability can be substantially mitigated by utilizing Hyperledger's enterprise-grade frameworks.
\clearpage 
%%%%%%%%%%%%%%%%%%%%%%%%%%%%%%%%%%%%%%%%%%%%%%%%%%%%%%%%%%%%
% PART IV – Proposed Model and Methodologies
%%%%%%%%%%%%%%%%%%%%%%%%%%%%%%%%%%%%%%%%%%%%%%%%%%%%%%%%%%%%
\part*{Part IV – Proposed Model and Methodologies}\addcontentsline{toc}{part}{Part IV – Proposed Model and Methodologies}

\section{Methodologies Utilized in Recent Papers}\label{sec:methods}
Methodologies in the recent literature typically involve simulation, prototype implementation, and formal ontology development. For instance, Khan et al. developed an ontology (CONIoT) to systematically classify consensus algorithms specifically for IoT environments \cite{khan2022asurveyand, khan2022asurveyand}. Simulation-based evaluations are common: papers report performance metrics such as latency, throughput, transaction confirmation time, and energy consumption in testbed implementations and numerical analyses \cite{ali2022blockchainenabledarchitecture, bryant2022keychallengesin}. Other studies incorporate deep learning and reinforcement learning to simulate adaptive consensus-particularly in the context of smart city applications-where a combination of blockchain and ML results in improved resource management and anomaly detection \cite{alam2023anoverviewof, guru2023asurveyon}. Each method is evaluated against typical IoT constraints, and simulation results are often validated by experimental prototypes in middleware frameworks like Hyperledger.

\section{Identified Limitations in Consensus Mechanism Research}\label{sec:limits}
Across the reviewed literature, several core limitations are frequently acknowledged:
\begin{itemize}
    \item High energy consumption and computational overhead inherent in PoW and many hybrid consensus models, rendering them impractical for IoT devices \cite{ali2022blockchainenabledarchitecture, khan2022asurveyand}.
    \item Scalability challenges as the network size increases, with protocols like PBFT suffering from O(n\textsuperscript{2}) communication complexity \cite{khan2022asurveyand, ali2022blockchainenabledarchitecture}.
    \item Latency issues in environments demanding near real-time processing, particularly when consensus protocols involve multiple validation rounds or heavy cryptographic calculations \cite{guo2022ahierarchicaland, morais2023surveyonintegration}.
    \item The complexity of dynamic consensus adaptation in heterogeneous networks, where frequent topology changes make stable consensus challenging \cite{alam2023anoverviewof, guru2023asurveyon}.
    \item Vulnerability to certain attacks, such as Sybil, if reputation systems are not robustly designed \cite{guru2023asurveyon, platt2023sybilinthe}.
    \item Limitations in fault tolerance due to reliance on trusted nodes or centralized components, which can reduce decentralization \cite{haque2024ascalableblockchain, morais2023surveyonintegration}.
\end{itemize}

These limitations, while varying with the type of consensus mechanism, underline the ongoing challenge of balancing security, efficiency, and resource constraints in IoT-specific blockchain deployments.

\section{Scalability, Interoperability, and Real-Time Data Processing}\label{sec:sir}
% Parent container (no text unless overview).
  \subsection{Scalability in Blockchain-Enabled IoT for Smart Agriculture}\label{subsec:scalability}
Scalability remains a major challenge for blockchain systems integrated with IoT in agriculture. The continuous generation of sensor data by hundreds of devices creates massive transaction loads that can exceed the capacity of conventional blockchain protocols. Traditional blockchains are optimized for applications with low-transaction frequencies, such as financial payments; in contrast, agricultural IoT systems may handle millions of transactions per day, leading to increased latency, energy consumption, and economic costs \cite{huang2025digitaltraceabilityin, irfan2025aniotdrivensmart}.

To address these issues, researchers have explored several strategies. Layer-2 scaling solutions such as state channels and sidechains allow noncritical or high-frequency data to be processed off-chain, thus reducing the main blockchain's load. Sharding techniques divide the network into smaller committees (``shards''), enabling each group of nodes to validate only a subset of transactions. One study demonstrated that with shards processing approximately 1,000 transactions per second each, the overall throughput can scale almost linearly with the number of shards deployed \cite{huang2025digitaltraceabilityin, abdurrohim2024blockchainbasedframeworkfor}. In addition, tailored consensus algorithms such as the Clustering and Reputation-based Practical Byzantine Fault Tolerance (CRPBFT) have been specifically designed for agricultural applications. CRPBFT has achieved latency reductions of around 73\% and energy savings up to 92\% compared with traditional Proof-of-Work protocols, thereby supporting high-frequency sensor data while meeting the low-power and cost constraints common in rural environments \cite{thiruvenkatasamy2025anonlinetool, huang2025digitaltraceabilityin}.

Selective anchoring is another approach that further optimizes scalability by committing only critical sensor events (for example, those that exceed predefined thresholds) to the blockchain. By reducing on-chain storage by as much as 95\%, selective anchoring maintains cryptographic verification of essential state changes without incurring the cost of recording every sensor reading \cite{huang2025digitaltraceabilityin, irfan2025aniotdrivensmart}. However, these strategies come with trade-offs: while they effectively improve throughput and lower latency, they add significant architectural complexity that must be balanced against energy consumption, administrative overhead, and economic viability in regions dominated by smallholder farmers \cite{huang2025digitaltraceabilityin, irfan2025aniotdrivensmart}.
  \subsection{Interoperability and Cross-Platform Solutions}\label{subsec:interoperability}
Interoperability represents a critical aspect of deploying blockchain-enabled IoT systems in smart agriculture because of the heterogeneous nature of the devices and data formats involved. In practice, agricultural IoT devices often come from different vendors and operate on proprietary communication standards, which can result in significant data fragmentation when integrating sensor information into a cohesive blockchain ledger \cite{irfan2025aniotdrivensmart, aliyu2023blockchainbasedsmartfarm}.

Several studies highlight the need for standardized communication protocols and semantic frameworks to ensure interoperability across these diverse systems. The adoption of open protocols such as MQTT, CoAP, and OPC-UA, combined with industry standards like ISO 11783 (ISOBUS) for farm machinery, facilitates a common language for data exchange \cite{irfan2025aniotdrivensmart, abdurrohim2024blockchainbasedframeworkfor}. Furthermore, semantic web technologies, particularly those based on RDF/OWL ontologies, have been employed to preserve the contextual fidelity of agricultural data during cross-chain transfers. By improving data contextualization up to 88\%, these frameworks help mitigate semantic degradation---reported to be as high as 23\% in some cases---thereby ensuring that nuanced information such as organic certification details is retained for precision applications \cite{irfan2025aniotdrivensmart, huang2025digitaltraceabilityin}.

Hybrid blockchain models that combine private and public chains have also been proposed as a means of addressing both interoperability and security. These architectures allow for low-latency processing via local private chains while leveraging public blockchains for transparency and auditability. A promising approach within this realm is the implementation of cross-chain atomic swap mechanisms based on hashed time-locked contracts (HTLCs), which have achieved asset transfer success rates exceeding 94\% and facilitate seamless value and data exchange across separate blockchain systems \cite{abdurrohim2024blockchainbasedframeworkfor, huang2025digitaltraceabilityin}. Hierarchical blockchain architectures further support interoperability by integrating local edge-level chains with a global public ledger, ensuring that disparate IoT devices and systems eventually reconcile into a unified, tamper-proof record \cite{thiruvenkatasamy2025anonlinetool, irfan2025aniotdrivensmart}.
  \subsection{Real-Time Data Processing}\label{subsec:realtime}
Real-time processing is a fundamental requirement in smart agriculture, where decisions related to irrigation, pest control, and crop management must be executed with minimal delay to ensure optimal production outcomes. The integration of blockchain with IoT, however, naturally introduces latency because blockchain consensus mechanisms, cryptographic computations, and data propagation delay inherently slow down transaction finality \cite{huang2025digitaltraceabilityin, huang2025digitaltraceabilityin}.

To overcome these challenges, a multi-layered approach combining edge computing, federated learning, and asynchronous consensus mechanisms has been proposed. Lightweight AI models deployed on edge devices can achieve inference speeds below 50 milliseconds, allowing for immediate local decisions in response to changes in environmental conditions \cite{huang2025digitaltraceabilityin, huang2025digitaltraceabilityin}. Federated learning frameworks enable these edge devices to train local AI models on sensitive data without the need to share raw data, while asynchronous updates allow for global consensus to be reached in the background without impeding the rapid local responses \cite{huang2025digitaltraceabilityin}.

Another strategy to achieve real-time performance involves off-chain processing, where bulk sensor data are stored externally (e.g., using the InterPlanetary File System, or IPFS) and only critical events are selectively anchored onto the blockchain. This approach significantly reduces the load on the blockchain, thereby decreasing confirmation times and enabling near real-time responsiveness for high-priority events \cite{huang2025digitaltraceabilityin, huang2025digitaltraceabilityin}. Moreover, emerging asynchronous consensus models and hierarchical validation schemes enable local consensus to be reached rapidly, triggering immediate interventions, while full global validation is achieved in parallel on a slower time scale \cite{huang2025digitaltraceabilityin, huang2025digitaltraceabilityin}.

\section{Proposed CRT-Based Parallel Transaction Model with Consensus and QoS}\label{sec:crt-model}
Despite the progress highlighted above, current blockchain-enabled IoT systems often struggle to reconcile the need for scalability, interoperability, and real-time processing within the dynamic and high-volume environment of smart agriculture. Our proposed solution introduces a CRT-based parallel transaction model that leverages the mathematical strength of the Chinese Remainder Theorem (CRT) to partition blockchain processing into multiple parallel streams.

In this novel model, incoming sensor data from diverse agricultural IoT devices are first partitioned into different segments (or shards) based on criteria such as sensor type, geographic region, and event criticality. Each shard functions as an independent microchain that processes a subset of transactions concurrently with other shards. This parallel processing dramatically reduces overall latency and enhances throughput because multiple consensus processes occur simultaneously rather than sequentially \cite{thiruvenkatasamy2025anonlinetool, hussein2024aiandiot}.

To ensure that the parallel streams remain fully synchronized and secure, our model integrates specialized consensus algorithms optimized for the specific characteristics of agricultural IoT workloads. For instance, adaptations of CRPBFT that have already demonstrated significant latency reductions and energy efficiencies are employed within each microchain, while periodic synchronization events ensure that the local states are reconciled onto a central public ledger. This layered consensus strategy maintains the global immutability and security of the blockchain while still supporting the high-volume, real-time demands of sensor data processing \cite{thiruvenkatasamy2025anonlinetool, huang2025digitaltraceabilityin}.

In addition, our model incorporates dynamic quality-of-service (QoS) mechanisms within its protocol stack. QoS parameters are used to prioritize transactions based on their criticality. For example, sensor events that indicate a sharp drop in soil moisture or a rapid onset of pest infestation are assigned higher priorities, ensuring that these events are processed with minimal delay compared with routine sensor updates. Such dynamic prioritization is achieved through adaptive communication channels and resource allocation protocols that adjust processing parameters in real time \cite{singh2025blockchainandflbased, thiruvenkatasamy2025anonlinetool}.

From an interoperability standpoint, the proposed CRT-based model enforces standardized data formats and utilizes semantic web frameworks (such as RDF/OWL) to maintain data context and integrity across parallel chains. The system is also designed to support cross-chain interactions through predefined communication protocols and structured data exchange formats, thereby ensuring that the heterogeneous data generated by diverse IoT devices can be seamlessly integrated into the global ledger \cite{irfan2025aniotdrivensmart, huang2025digitaltraceabilityin}.

\subsection{Mathematical Foundation: Chinese Remainder Theorem (CRT)}
\label{sec:CRT}

Let $m_1,\dots,m_k\in\mathbb{Z}_{>0}$ be pairwise coprime ($\gcd(m_i,m_j)=1$ for $i\neq j$) and let $M=\prod_{i=1}^k m_i$.
For any residue vector $(a_1,\dots,a_k)$ with $0\le a_i<m_i$, the system
\[
x \equiv a_i \pmod{m_i}\quad (i=1,\dots,k)
\]
admits a unique solution modulo $M$.
A constructive solution is obtained by defining $M_i = M/m_i$ and $u_i\equiv M_i^{-1}\pmod{m_i}$ (the modular inverse).
Then a canonical representative is
\[
x \equiv \sum_{i=1}^k a_i \, M_i \, u_i \;\;(\mathrm{mod}\; M).
\]
\textbf{Interpretation for parallel chains.} Map accounts/keys (or transaction buckets) to residue classes $\bmod\,m_i$. Each microchain $\mathcal{S}_i$ processes transactions assigned to class $i$ independently. Periodically, we form a global checkpoint by reconstructing an aggregate state index (e.g., counter, epoch root, or sequence number) via the CRT formula, yielding a single canonical value modulo $M$ that is consistent with all shards. This provides a mathematically clean merge primitive while preserving per‑shard concurrency.

\paragraph{Correctness invariant.}
If each $\mathcal{S}_i$ is consistent with its local order, then the merged index $x\pmod{M}$ reconstructed by CRT is unique. Thus any two honest verifiers reconstructing from the same per‑shard residues obtain the same global value, ensuring determinism of merges.

\subsection{Performance Metrics for Parallel Transaction Streams}
\label{sec:crt-perf}

Assume shard $i$ receives Poisson arrivals with rate $\lambda_i$ and has an effective service rate $\mu_i$ (consensus + execution). For an M/M/1 approximation:
\[
\text{stability: } \rho_i=\lambda_i/\mu_i<1,\qquad \mathbb{E}[W_i] = \frac{1}{\mu_i-\lambda_i},
\]
where $W_i$ is the mean sojourn time (queueing + service). The system throughput is
\[
T_{\mathrm{total}} = \sum_{i=1}^k \min\{\lambda_i,\mu_i\},
\]
and the end‑to‑end latency for a batch merge is approximated by
\[
L_{\mathrm{e2e}} \approx \max_{i}\big(W_i + C_i\big) + \tau_m,
\]
where $C_i$ is the commit time on shard $i$ and $\tau_m$ is the merge time (CRT reconstruction + cross‑shard verification).

\paragraph{Merge cost.}
Let $b=\lceil\log_2 M\rceil$ denote the bit‑length of $M$. A CRT merge requires $k$ modular inversions (precomputable) and $k$ multiplications on $b$‑bit integers. Using schoolbook arithmetic this is $O(k\,b^2)$ time; faster methods reduce this to near‑linear in $k$ for moderate $b$. We model it as
\[
\tau_m \approx \alpha\,k\,b^{\gamma} + \beta\,k,
\]
where $\alpha,\beta,\gamma$ are hardware‑dependent constants (empirically $\gamma\in[1,2]$). Increasing $k$ accelerates throughput until $\tau_m$ and cross‑shard dependencies dominate; this provides a principled tuning knob.

\noindent\textit{Link to QoS.} Since $C_i$ and $\mu_i$ are protocol‑dependent, consensus selection directly shapes $\{W_i\}$ and hence $L_{\mathrm{e2e}}$ and $T_{\mathrm{total}}$ (see §\ref{sec:consensus-qos}).

% Put this where the image should appear
\begin{figure}[htbp]
  \centering
  % adjust the path if liot.png lives in e.g., images/liot.png
  \includegraphics[width=\linewidth]{liot.png}
  \caption{Schematic of the LIoT pipeline~\cite{thiruvenkatasamy2025anonlinetool}.}
  \label{fig:liot}
\end{figure}

As shown in \autoref{fig:liot}, the system is organized schematically illustrates the CRT-based parallel transaction architecture. In the figure, sensor data streams are first segregated into distinct shards based on type and priority, processed concurrently via local consensus mechanisms, and then periodically aggregated to update the public blockchain. This dual-layer architecture not only reduces latency but also provides dynamic QoS by adapting resource allocation based on real-time sensor data and pre-configured priority thresholds \cite{thiruvenkatasamy2025anonlinetool, huang2025digitaltraceabilityin}.

\begin{table}[htbp]
\caption{Summary of Selected Papers on Scalability and Real-Time Processing}
\label{tab:scalability}
\centering
\fitToPage{
\begingroup
\setlength{\tabcolsep}{4pt}
\renewcommand{\arraystretch}{1.15}
\footnotesize
\begin{tabularx}{\textwidth}{@{}P{0.18\textwidth}YYY@{}}
\toprule
\textbf{Paper} & \textbf{Key Findings} & \textbf{Methodologies} & \textbf{Limitations} \\
\midrule
Huang et al. (2025) & Demonstrated effective use of Layer-2 protocols and sharding to enhance throughput; increased TPS from $\sim$15 to over 10,000 & Implementation of Layer-2 protocols and sharding techniques for blockchain scalability & Evaluated under limited testbed conditions with small node clusters \\
\addlinespace
Huang et al. (2025) & Integrated edge computing with blockchain consensus achieving sub-second latency; reported up to 85\% latency reduction & Edge computing integration with lightweight AI algorithms for latency reduction & Tested in small-scale deployments; energy consumption issues persist \\
\addlinespace
Thiruvenkatasamy et al. (2025) & Achieved 73\% latency reduction using CRPBFT and selective anchoring strategies that reduced on-chain data volume by 95\% & CRPBFT consensus with selective anchoring strategies for data volume reduction & Complexity in coordinating consensus across multiple tiers under dynamic conditions \\
\addlinespace
Huang et al. (2025) & Enabled near real-time edge inference ($<$50 ms) via federated learning and asynchronous verification; achieved overall prediction accuracy $>$98.6\% & Federated learning with asynchronous verification for real-time edge inference & Trade-off between rapid local decisions and delayed global verification \\
\bottomrule
\end{tabularx}
\endgroup
}
\end{table}

\begin{table}[htbp]
\caption{Summary of Selected Papers on Interoperability and Cross-Platform Solutions}
\label{tab:interoperability}
\centering
\fitToPage{
\begingroup
\setlength{\tabcolsep}{4pt}
\renewcommand{\arraystretch}{1.15}
\footnotesize
\begin{tabularx}{\textwidth}{@{}P{0.18\textwidth}YYY@{}}
\toprule
\textbf{Paper} & \textbf{Key Findings} & \textbf{Methodologies} & \textbf{Limitations} \\
\midrule
Irfan et al. (2025) & Improved metadata contextual fidelity to up to 88\% using RDF/OWL frameworks; reduced semantic degradation to 77\% during cross-chain data transitions & RDF/OWL frameworks for semantic interoperability in cross-chain data transitions & Requires further standardization and broader empirical validation \\
\addlinespace
Abdurrohim et al. (2024) & Utilized HTLC-based mechanisms to achieve asset transfer success rates $>$94\% and enable secure cross-chain transactions & HTLC-based mechanisms for secure cross-chain transactions & Increased computational overhead and energy demands for cryptographic operations \\
\addlinespace
Huang et al. (2025) & Demonstrated that edge-level private chains periodically anchoring to a public chain can reduce infrastructure costs by 73\% while ensuring interoperability & Hierarchical blockchain structures with periodic anchoring to public chains & Requires complex synchronization across network tiers under variable conditions \\
\addlinespace
Irfan et al. (2025) & Advocated for the adoption of open standards (MQTT, CoAP, OPC-UA) to facilitate seamless interoperability across heterogeneous devices & Implementation of standardized IoT communication protocols (MQTT, CoAP, OPC-UA) & Limited by legacy systems and varying levels of industry adoption \\
\bottomrule
\end{tabularx}
\endgroup
}
\end{table}

%%%%%%%%%%%%%%%%%%%%%%%%%%%%%%%%%%%%%%%%%%%%%%%%%%%%%%%%%%%%
% PART V – Critical Discussion and Future Outlook
%%%%%%%%%%%%%%%%%%%%%%%%%%%%%%%%%%%%%%%%%%%%%%%%%%%%%%%%%%%%
\part*{Part V – Critical Discussion and Future Outlook}\addcontentsline{toc}{part}{Part V – Critical Discussion and Future Outlook}

\section{Enhanced Report Discussion}\label{sec:enhanced-discussion}
\noindent This section synthesizes and critically analyzes the current state of blockchain-IoT integration for smart agriculture, as presented in recent literature (2022-2025). It consolidates findings on architectural designs, consensus mechanisms, performance trade-offs, and prevailing limitations, providing a comprehensive overview of the field's achievements and persistent challenges.

% --- BEGIN: Enhanced base content from XII ---
The advancement of IoT architectures for smart agriculture necessitates a holistic design philosophy that marries robust, scalable software frameworks with resilient and interoperable hardware. A critical analysis of the literature reveals a dominant trend towards multi-tiered architectures that strategically distribute computational load across the IoT-edge-fog-cloud continuum. Studies by \cite{akhter2022precisionagricultureusing} and \cite{atalla2023iotenabledprecisionagriculture} demonstrate that such hierarchical designs successfully mitigate latency and bandwidth limitations inherent in pure cloud-centric models. For instance, processing data at the edge or fog layer for local decision-making (e.g., immediate irrigation triggers) before committing verified results to the cloud for historical analysis and blockchain storage is a commonly advocated and effective pattern.

However, a significant gap persists between simulated validation and real-world deployment. A substantial portion of the proposed architectures, as noted by \cite{bayih2022utilizationofinternet} and \cite{atalla2023iotenabledprecisionagriculture}, are evaluated in controlled environments or simulations (e.g., COOJA simulator) that fail to capture the full spectrum of environmental dynamics, network instability, and resource contention found in operational farms. This simulation-reliance represents a critical validity threat to the claimed performance metrics.

Furthermore, interoperability remains a formidable obstacle. The agricultural IoT landscape is characterized by extreme heterogeneity in devices, protocols, and data formats. Research consistently highlights that this fragmentation \cite{bayih2022utilizationofinternet, bayih2022utilizationofinternet} severely hinders the creation of a unified data ecosystem, which is a prerequisite for advanced, cross-field analytics and decision-support systems. This lack of standardization often results in data silos and limits the scalability of proposed solutions.

The integration of AI and machine learning, particularly for predictive analytics and anomaly detection, is another prominent theme. Works by \cite{bakthavatchalam2022iotframeworkfor} and \cite{ouafiq2022datamanagementand} show enhanced capabilities in areas like disease prediction and yield forecasting. Yet, this integration introduces its own set of challenges, notably the significant computational and energy demands of running sophisticated models, which often clash with the resource-constrained nature of edge devices and sensor nodes. This necessitates efficient model compression techniques and a careful balancing of accuracy against computational cost.

In this context, blockchain technology emerges not as a panacea, but as a powerful enabler for specific trust and transparency problems. Early investigations into blockchain-enabled IoT frameworks \cite{quy2022iotenabledsmartagriculture, abunadi2022trafficawaresecuredcooperative} demonstrate its potential to secure data transactions, create immutable audit trails, and facilitate automated trust through smart contracts. The core value proposition lies in blockchain's ability to provide a single source of truth in decentralized, multi-stakeholder environments like agricultural supply chains.

When mapping these limitations to technological solutions, Hyperledger's permissioned blockchain model presents a particularly strong fit. Its features directly address several identified pain points:
\begin{itemize}
    \item \textbf{Data Authenticity \& Integrity:} Hyperledger's immutable ledger and cryptographic hashing provide a verifiable record of sensor data and supply chain events, mitigating risks of tampering and fraud.
    \item \textbf{Scalability via Channels:} The channel feature allows for the creation of sub-networks, enabling parallel processing of transactions for different operations (e.g., one channel for field sensor data, another for financial transactions with distributors), thereby alleviating throughput bottlenecks.
    \item \textbf{Interoperability via Standardization:} Hyperledger's well-defined APIs and support for standardized data formats through chaincode (smart contracts) can enforce consistency across heterogeneous device data, acting as a unifying layer.
    \item \textbf{Fault Tolerance:} Its modular consensus protocols (e.g., Raft) offer crash fault tolerance, enhancing the reliability of the overall system compared to centralized data handlers.
\end{itemize}

This approach is further reinforced by research suggesting that blockchain's inherent properties of verifiability and auditability can help bridge the gap between simulation and real-world validation by providing tamper-proof evidence of system performance under real conditions \cite{abunadi2022trafficawaresecuredcooperative, ouafiq2022datamanagementand}.

Energy consumption remains a paramount concern, especially for IoT deployments in remote areas with limited power infrastructure. While studies on self-powered systems \cite{raju2022aselfpoweredrealtime} show promise for individual devices, they often lack scalability to network-wide deployments. Here, the choice of blockchain platform becomes critical. Permissioned blockchains like Hyperledger, which utilize efficient consensus mechanisms like Raft or PBFT, inherently consume orders of magnitude less energy than the Proof-of-Work (PoW) systems often used as a baseline in criticism, making them far more suitable for sustainable agricultural applications.

In summary, the current state-of-the-art in IoT architectures for agriculture demonstrates significant theoretical potential and pilot-level success. The transition to widespread adoption, however, is hampered by the triad of challenges: (1) the simulation-to-reality gap, (2) device and protocol heterogeneity, and (3) the energy-performance trade-off. The integration of enterprise-grade permissioned blockchain frameworks, specifically designed for modularity and efficiency, is identified as a promising pathway to systematically address these barriers by enhancing data authenticity, security, and system-level interoperability.
% --- END: Enhanced base content from XII ---

% --- BEGIN: Enhanced general discussion from XIX ---
Building upon the architectural analysis, the literature collectively affirms the immense potential of blockchain-IoT integration for revolutionizing smart agriculture. The core value is consistently demonstrated in enhancing data security, ensuring transparent traceability from farm to fork, and enabling automated, trustless execution of agreements via smart contracts \cite{aliyu2023blockchainbasedsmartfarm, ellahi2023blockchainbasedframeworksfor}.

A granular examination of consensus mechanisms reveals a critical performance bottleneck. While numerous studies propose lightweight alternatives to PoW, many consensus protocols, including Practical Byzantine Fault Tolerance (PBFT) and its variants, remain insufficiently optimized for the specific demands of agricultural IoT networks. These networks are characterized by their resource constraints, intermittent connectivity, and dynamic topology. The computational overhead and communication complexity (often O(n²)) of these protocols lead to increased latency and reduced throughput, as documented by \cite{ali2022blockchainenabledarchitecture} and \cite{morais2023surveyonintegration}. This creates a direct conflict with the low-latency requirements of real-time agricultural decision-making, such as precision irrigation or automated pest control.

Similarly, proposed solutions involving off-chain processing and edge computing, while theoretically sound for improving Quality of Service (QoS), introduce their own integration complexities. The literature points to significant challenges in creating seamless, secure, and efficient data pipelines between IoT sensors, edge processing units, and the blockchain layer. The work of \cite{khan2022ablockchainand} and \cite{sizan2505asecuredtriad} highlights that managing data consistency, security policy enforcement, and transaction finality across these heterogeneous layers remains a non-trivial engineering problem.

Within this landscape of challenges, Hyperledger Fabric is analytically positioned as a viable and strategic solution. Its architectural advantages directly counter the identified limitations:
\begin{enumerate}
    \item \textbf{Permissioned Architecture:} By eliminating the need for energy-intensive mining, Hyperledger drastically reduces the computational overhead and energy consumption, making it inherently compatible with sustainable agricultural operations and resource-constrained environments.
    \item \textbf{Modular Consensus:} The pluggable consensus model allows the system to be tailored to specific network conditions and performance requirements. A network can be configured with Raft for low-latency, high-throughput environments within a single farm or organization, while other BFT-like protocols could be explored for broader consortia, providing unparalleled flexibility to address the latency-throughput trade-off.
    \item \textbf{Channel-Based Partitioning:} The ability to create private channels is a powerful tool for scalability and privacy. Critical, time-sensitive data (e.g., actuator commands) can be processed on a dedicated high-priority channel, while less critical telemetry data is handled on another. This effectively isolates traffic and prevents network congestion, directly enhancing QoS.
    \item \textbf{Enhanced Interoperability and Security:} Hyperledger's Membership Service Provider (MSP) provides robust identity management, which is crucial for authenticating diverse IoT devices and users within the network. Its well-documented APIs facilitate integration with existing legacy systems and heterogeneous IoT platforms, addressing a key interoperability hurdle noted across the literature \cite{aliyu2023blockchainbasedsmartfarm, tang2024assessingblockchainand}.
\end{enumerate}
% --- END: Enhanced general discussion from XIX ---

% --- BEGIN: Enhanced general discussion from XXXI ---
Delving into the domains of security, privacy, and reliability, the literature underscores a strategic shift towards more sophisticated, integrated architectural paradigms. The consensus is clear: blockchain's decentralized trust model provides a foundational advantage over centralized systems by eliminating single points of failure and creating cryptographically verifiable data histories \cite{aliyu2023blockchainbasedsmartfarm, soy2025blockchainintegrationin}.

A key trend identified is the migration towards \textbf{permissioned blockchain systems}, notably Hyperledger Fabric and Corda. This shift is driven by the need for greater control over network participation, enhanced privacy through data partitioning (channels), and superior performance—attributes that are essential for business-centric agricultural applications. These platforms facilitate smart contract-driven automation for processes like payments, quality certification, and compliance reporting, directly addressing the operational inefficiencies highlighted in numerous studies.

Concurrently, there is a growing emphasis on \textbf{privacy-preserving computation}. The integration of techniques like federated learning \cite{daund2025designofan} allows for the collaborative training of machine learning models on decentralized data without exposing raw, sensitive farm information. This is often combined with differential privacy mechanisms to further protect against inference attacks. This approach represents a significant advancement over simply storing encrypted data on-chain, as it enables value extraction from data while preserving confidentiality.

However, this synthesis also exposes several persistent and intertwined research gaps:
\begin{itemize}
    \item \textbf{Scalability vs. Resource Constraints:} While proposals like sharding and sidechains show promise for increasing transaction throughput, their practical implementation on networks of resource-constrained agricultural IoT devices remains largely unproven. The energy and computational overhead of maintaining multiple shards or processing off-chain transactions can be prohibitive, creating a tension between scalability and sustainability.
    \item \textbf{The Transparency-Privacy Paradox:} A fundamental tension exists between blockchain's core principle of transparency and the legitimate need for data privacy in business operations. Emerging hybrid models that combine private execution with public verification (e.g., using zero-knowledge proofs) are promising but are often complex, computationally expensive, and not yet matured for large-scale agricultural deployment \cite{soy2025blockchainintegrationin}.
    \item \textbf{Integration Complexity:} The incorporation of advanced cryptographic techniques (e.g., homomorphic encryption, ZKPs) and AI methods, while enhancing security and functionality, adds significant layers of complexity to the system architecture. This complexity manifests in increased development difficulty, higher computational costs, and potential vulnerabilities in the interactions between these complex components.
\end{itemize}

The proposed CRT-based parallel transaction model, referenced in the literature, is an example of an attempt to address the QoS and scalability challenge. By leveraging mathematical principles to partition transaction processing, it aims to achieve higher throughput and lower latency. However, its integration with consensus mechanisms and its behavior under the heavy, bursty data loads typical of agricultural IoT networks (e.g., during harvest or weather events) require thorough investigation. Future research must focus not only on these individual mechanisms but also on \textbf{adaptive frameworks} that can dynamically tune security protocols, resource allocation, and consensus parameters in response to real-time network conditions and transaction criticality.

Ultimately, the literature points to the need for \textbf{holistic co-design}. The most promising path forward is not to optimize security, privacy, and performance in isolation but to develop integrated frameworks where these concerns are addressed concurrently. This involves the co-design of lightweight cryptographic protocols, energy-aware consensus algorithms, and efficient data management strategies that are purpose-built for the unique constraints and requirements of agricultural ecosystems.
% --- END: Enhanced general discussion from XXXI ---

The literature review reveals a concerted effort in recent years to develop or adapt consensus protocols that satisfy the particular demands of IoT environments. The integration of hierarchical positioning, selective consensus decisions, and adaptive machine learning techniques form the core of many proposals \cite{ali2022blockchainenabledarchitecture, guru2023asurveyon, guo2022ahierarchicaland}. Nevertheless, each approach tends to encounter similar limitations-specifically, high energy consumption, difficulty in scaling to networks with large numbers of resource-constrained devices, and increased communication overhead. Such limitations are further compounded by security issues that arise due to low computational power available for implementing robust cryptographic measures. Importantly, many proposals require complex architectures that may not be easily standardized across diverse IoT deployments.

In contrast, Hyperledger Fabric and similar Hyperledger projects have demonstrated maturity in addressing many of these issues through permissioned blockchain configurations that optimize resource utilization, maintain decentralization via endorsement policies, and provide flexible consensus options that can be dynamically adjusted \cite{bryant2022keychallengesin, khan2022asurveyand}. As a result, Hyperledger emerges as a potential baseline for IoT applications in smart agriculture - where selective and parallel consensus, as well as QoS mechanisms, can be built on a foundation that is already optimized for secure, efficient, and scalable blockchain solutions.

The state-of-the-art review reveals significant progress in the integration of blockchain and IoT for smart agriculture, yet also underscores critical challenges that remain unsolved. From an energy efficiency standpoint, researchers have increasingly focused on developing lightweight blockchain architectures and alternative consensus mechanisms that can be deployed on battery-powered IoT devices. Studies such as Munaganuri et al. \cite{munaganuri2025designofan} demonstrate that combining low-power communication protocols like LoRaWAN with optimized blockchain implementations can yield meaningful reductions in energy consumption. Nonetheless, many proposals are still limited by scalability issues and the computational overhead of even lightweight cryptographic operations, particularly when applied to large-scale, heterogeneous agricultural environments \cite{akella2023asystematicreview, munaganuri2025designofan}. Although certain approaches have shown promise in simulated environments and small-scale deployments, further validation is required to ensure these techniques can be effectively extrapolated to broader agricultural contexts while maintaining energy efficiency \cite{tahayur2024enhancingelectronicagriculture, mwewa2024blockchaintechnologya}.



Blockchain-enabled traceability systems hold enormous potential for transforming the agricultural supply chain by providing absolute data integrity, enabling real-time tracking, and automating transactions through smart contracts. Numerous case studies have demonstrated the feasibility of these systems in reducing food fraud and enhancing consumer trust by providing immutable records of product provenance \cite{akella2023asystematicreview, mwewa2024blockchaintechnologya}. Yet, the extant literature also highlights that many implementations are fragmented, catering only to specific segments of the supply chain rather than offering a full-stack solution that bridges all stages of production, processing, and distribution. Moreover, interoperability issues persist, as many blockchain frameworks are not yet seamlessly integrated with legacy agricultural ICT systems, thereby limiting their scalability and overall effectiveness \cite{akella2023asystematicreview}.

\section*{Mapping Research Limitations to Hyperledger Solutions}

A recurring theme in the reviewed literature is that many of the observed limitations in blockchain-enabled IoT systems for smart agriculture can be addressable, at least in part, by the capabilities inherent to Hyperledger platforms. In this section, we map several common limitations reported by researchers to the corresponding solutions offered by Hyperledger.

\begin{enumerate}
\item \textbf{Scalability and Throughput}  
   Several studies \cite{ali2022blockchainenabledarchitecture, demestichas2020blockchaininagriculture} note that traditional consensus mechanisms fail to provide the required throughput when faced with high volumes of IoT-generated data. Hyperledger Fabric mitigates these issues through its channel-based architecture that allows parallel processing of transactions and modular consensus plug-ins (e.g., Raft or Kafka) that can be tailored to meet dynamic workloads.

\item \textbf{Consensus Efficiency and Adaptability}  
   The selective consensus selection approaches \cite{ali2022blockchainenabledarchitecture} propose dynamically changing the consensus protocol based on network conditions. Hyperledger Fabric's flexible design supports pluggable consensus protocols in permissioned environments, thereby reducing latency and computational overhead and adapting consensus to current network conditions.

\item \textbf{Resource Constraints of IoT Devices}  
   The limited computing and storage capacities of IoT devices, as frequently noted \cite{ali2022blockchainenabledarchitecture, aliyu2023blockchainbasedsmartfarm}, are addressed in Hyperledger Fabric by offloading heavy computational tasks to endorsing nodes. Lightweight clients required on IoT devices need only maintain minimal state (e.g., block headers) and perform API calls, thus preserving the energy and processing resources of edge devices.

\item \textbf{Data Heterogeneity and Integration Challenges}  
   Multiple studies \cite{bosona2023theroleof, chandan2023achievingunsdgs} identify difficulties in standardizing data from heterogeneous sources across complex supply chains. Hyperledger Fabric's support for customizable chaincode (smart contracts) enables operators to enforce uniform data formats and validation rules, thereby facilitating integration among disparate legacy systems and modern IoT devices.

\item \textbf{Security, Privacy, and Quality-of-Service (QoS)}  
   Blockchain-based systems are consistently challenged by ensuring high QoS in the face of large data volumes and cyber threats \cite{aliyu2023blockchainbasedsmartfarm, demestichas2020blockchaininagriculture}. Hyperledger Fabric enhances security via robust identity management, role-based access control, and private channels that segregate sensitive transactions, thus ensuring both improved QoS and enhanced data privacy.

\item \textbf{Parallel Transaction Processing}  
   Emerging solutions that implement CRT-based parallel transaction models to improve throughput \cite{ali2022blockchainenabledarchitecture} are conceptually similar to Hyperledger Fabric's ordering service architecture, which supports parallel transaction execution across multiple channels. This effectively reduces processing bottlenecks and improves the reliability of real-time applications in agricultural environments.
\end{enumerate}



\section{Critical Discussion and Future Directions}\label{sec:crit-future}
The literature reviewed herein collectively indicates that blockchain-enabled IoT frameworks offer immense potential for smart agriculture by ensuring secure data management, traceability, and automated decision-making through smart contracts \cite{aliyu2023blockchainbasedsmartfarm, ellahi2023blockchainbasedframeworksfor}. Nonetheless, several critical challenges have hindered the practical deployment of these systems in large-scale, heterogeneous agricultural environments. In many cases, the underlying consensus protocols are not adequately optimized for energy- and resource-constrained devices, leading to increased latency and throughput limitations \cite{ali2022blockchainenabledarchitecture, morais2023surveyonintegration}. Similarly, although off-chain processing and edge computing promise enhancements in QoS, significant integration issues persist between disparate IoT sensors and the blockchain layer itself \cite{khan2022ablockchainand, sizan2505asecuredtriad}.

Hyperledger Fabric presents a viable solution to many of these issues. First, its permissioned architecture greatly reduces the unnecessary overhead associated with proof-of-work consensus, making it more suited to the closed, controlled environments common in agricultural applications. Moreover, by allowing for customizable consensus protocols through modular design, Hyperledger can be tuned to prioritize low latency and high throughput, directly addressing the performance deficits observed in the reviewed works \cite{ellahi2023blockchainbasedframeworksfor, sajja2023towardsapplicabilityof}. Second, the ability to partition the network into channels and use private data collections not only secures sensitive agricultural data but also improves scalability as fewer nodes process sensitive transactions \cite{sakthivel2024enhancingtransparencyand, morais2023surveyonintegration}. Third, Hyperledger's extensive support for interoperability and integration—through its Membership Service Provider and well-documented APIs—helps overcome the challenges associated with integrating heterogeneous IoT sensor networks and legacy systems \cite{aliyu2023blockchainbasedsmartfarm, tang2024assessingblockchainand}. Finally, the Hyperledger ecosystem's focus on industry standards and continuous updates provides a roadmap towards mitigating many of the shortfalls that remain in existing research prototypes.

Going forward, future research should focus on real-world pilot deployments of blockchain-enabled IoT frameworks in diverse agricultural settings to rigorously evaluate the performance of Hyperledger-based solutions under highly variable conditions. In particular, adaptive consensus tuning (e.g., dynamic Raft configuration based on network load), advanced edge computing strategies, and integrated AI-driven predictive analytics should be systematically explored. In addition, research must also address legal, regulatory, and interoperability challenges through cross-industry collaborations, ensuring that technological advances in blockchain and IoT translate into sustainable, cost-effective solutions for smart agriculture \cite{bodkhe2022blockchainforprecision, yakubu2022ricechainsecureand}.

The collected literature highlights significant advancements in the convergence of blockchain with IoT for smart agriculture, particularly regarding security, privacy, and reliability. A common observation is that blockchain's decentralized architecture provides a solid foundation by eliminating centralized vulnerabilities and establishing immutable transaction records. However, each proposed framework must strike a delicate balance between implementing robust security protocols and meeting the resource constraints of agricultural IoT devices.

Notably, many state-of-the-art frameworks have shifted toward employing permissioned blockchain systems (such as Hyperledger Fabric and Corda) to enhance access control and enable smart contract-driven automation \cite{aliyu2023blockchainbasedsmartfarm, soy2025blockchainintegrationin}. At the same time, privacy-centric approaches are increasingly integrating federated learning and differential privacy to protect sensitive data during distributed analytics \cite{daund2025designofan}. These integrated solutions illustrate a trend toward holistic architectures that address both external cyber threats (e.g., poisoning, man-in-the-middle, and DDoS attacks) and internal vulnerabilities related to data leakage and unauthorized access.

Despite these promising developments, several open research gaps remain. First, scalability is a universal challenge across many proposals. While experiments in relatively controlled environments indicate improved throughput and fault tolerance, the transition to large-scale deployments in varied agricultural settings has not been fully validated. Second, the resource consumption associated with advanced encryption, consensus algorithms, and redaction methods may exceed the capabilities of resource-constrained edge devices typical in rural farms. Addressing such limitations requires further exploration of lightweight cryptographic methods and energy-efficient consensus mechanisms. Third, the inherent tension between maintaining blockchain transparency and ensuring robust privacy protection is yet to be completely resolved; emerging hybrid models that combine public and private blockchain features show promise but need further robustness evaluation \cite{soy2025blockchainintegrationin}.

Moreover, the integration of Quality-of-Service (QoS) parameters with consensus processes—such as the proposed CRT-based parallel transaction model—introduces additional layers of complexity. Future research should explore adaptive frameworks capable of dynamically adjusting security protocols and resource allocation based on network conditions and transaction criticality. In this context, additional evaluations on latency, energy consumption, and throughput across heterogeneous IoT platforms are warranted.

Finally, while many studies focus on isolated aspects of security, privacy, and reliability, there is a growing need for comprehensive frameworks that simultaneously address these interconnected domains. The emerging trend of combining blockchain with advanced AI methods (e.g., deep learning for anomaly detection) and cryptographic techniques (e.g., homomorphic encryption, ZKP) points toward integrated solutions that can adapt to the dynamic and heterogeneous nature of agricultural data flows. Such systems would not only secure data but also provide actionable insights to optimize agricultural operations and ensure sustainability.




\section{State-of-the-Art Comparison and Discussion}\label{sec:soa-discussion}
Comparing the recent literature reveals a convergence toward integrating blockchain with advanced cryptographic and machine learning techniques to solve persistent problems in smart agriculture. Works by Aliyu and Liu set the stage by demonstrating practical implementations that focus on poisoning attack prevention and secure IoT data logging using Ethereum and cloud infrastructures. In parallel, studies such as those by Daund et al. have emphasized the incorporation of federated learning and differential privacy to enhance threat detection while maintaining data confidentiality. Similarly, research focused on fault-tolerant architectures shows that multi-tiered blockchain designs not only improve reliability but also enable scalability and real-time responsiveness under resource-limited conditions.

A key area of convergence across these works is the use of permissioned blockchains to tailor network access and provide controlled transparency in agricultural supply chains. While Ethereum's public blockchain offers versatility, its limitations in scalability and transaction costs have led researchers to explore alternatives like Hyperledger Fabric, which offer modular designs and improved privacy controls \cite{aliyu2023blockchainbasedsmartfarm, soy2025blockchainintegrationin}. In addition, the integration of advanced cryptographic primitives such as ECC, zero-knowledge proofs, and homomorphic encryption is becoming common practice to secure sensitive data without impeding computational performance.

In terms of limitations, several studies note that although the integration of blockchain with advanced security and privacy mechanisms has produced promising results in controlled prototype environments, there remains a significant gap in large-scale practical deployments. Issues such as computational overhead, battery life and energy consumption issues for IoT sensors, and integration complexities with existing agricultural infrastructures continue to be open research challenges. Furthermore, efforts to balance the inherent transparency of blockchain with stringent privacy requirements have led to hybrid solutions that require further optimization to minimize trade-offs between data confidentiality and auditability \cite{soy2025blockchainintegrationin}.

% Figure 1
\begin{figure}[htbp]
  \centering
  \includegraphics[width=\linewidth]{fig1.png}
  \caption{Smart farming application in cloud-based IoT, from Aliyu et al. (2023)~\cite{aliyu2023blockchainbasedsmartfarm}.}
  \label{fig:fig1}
\end{figure}

% Figure 2
\begin{figure}[htbp]
  \centering
  \includegraphics[width=\linewidth]{fig2.png}
  \caption{Taxonomy of blockchain-based frameworks, from Ellahi et al. (2023)~\cite{ellahi2023blockchainbasedframeworksfor}.}
  \label{fig:fig2}
\end{figure}

% Text referencing Figures
Figure~\ref{fig:fig1} illustrates an overview of a typical multi-tiered blockchain-enabled smart agriculture framework, highlighting the roles of edge, fog, and cloud layers in data collection, processing, and secure storage.  Figure~\ref{fig:fig2} depicts a sample smart contract workflow used for real-time threat detection and automated alerting in a smart farm setting.


Below are two summary tables capturing groups of similar studies on blockchain-based security, privacy, and reliability in smart agriculture.
\begin{table}[H]
\caption{Summary of Papers Focusing on Security and Cryptographic Techniques}
\label{tab:security}
\centering
\fitToPage{
\begingroup
\setlength{\tabcolsep}{4pt}
\renewcommand{\arraystretch}{1.15}
\footnotesize
\begin{tabularx}{\textwidth}{@{}P{0.18\textwidth}YYY@{}}
\toprule
\textbf{Paper} & \textbf{Key Findings} & \textbf{Methodologies} & \textbf{Limitations} \\
\midrule
Aliyu and Liu (2023) & Demonstrates blockchain for tamper-proof IoT sensor data logging and poisoning attack prevention & Integration of AWS IoT with Ethereum smart contracts for secure data logging & Limited experimental runs; reliance on cloud infrastructure may impact latency and scalability \\
\addlinespace
Daund et al. (2025) & Integrates blockchain with federated learning and differential privacy, ensuring 100\% data integrity & Hyperledger Fabric with federated learning and differential privacy techniques for secure ML threat detection & Computational overhead and integration complexity; scalability challenges in resource-limited IoT systems \\
\addlinespace
Mishra et al. (2023) & Proposes a three-tiered architecture with redactable blockchain for secure data aggregation & Fog-enabled agricultural IoT network with redactable blockchain for source authentication and tamper resistance & Resource limitations at fog nodes; challenges in integrating advanced redaction mechanisms \\
\addlinespace
Tahayur et al. (2024) & Employs advanced cryptographic techniques for strong access control and device authentication & Elliptic Curve Cryptography (ECC) and ring signatures for IoT device authentication & Higher computational cost for cryptographic operations; potential latency issues in real-time processing \\
\addlinespace
Tahayur et al. (2024) & Investigates scalable, resource-efficient blockchain models for agriculture IoT & Smart contracts on permissioned blockchains optimized for resource-constrained environments & Trade-offs between security robustness and energy consumption; latency limitations persist \\
\bottomrule
\end{tabularx}
\endgroup
}
\end{table}

\begin{table}[H]
\caption{Summary of Papers Focusing on Privacy Preservation and Reliability}
\label{tab:privacy}
\centering
\fitToPage{
\begingroup
\setlength{\tabcolsep}{4pt}
\renewcommand{\arraystretch}{1.15}
\footnotesize
\begin{tabularx}{\textwidth}{@{}P{0.18\textwidth}YYY@{}}
\toprule
\textbf{Paper} & \textbf{Key Findings} & \textbf{Methodologies} & \textbf{Limitations} \\
\midrule
Rahaman et al. (2024) & Proposes a three-phase secure data exchange protocol for rural farm monitoring & Symmetric/asymmetric encryption, hash functions, and rigorous authentication protocols & Scalability issues in large-scale deployments; limited evaluation of real-world impact \\
\addlinespace
Daund et al. (2025) & Combines blockchain with federated learning and noise addition to protect sensitive IoT data & Federated learning with differential privacy and noise addition techniques; blockchain for integrity & Trade-offs between privacy guarantees and computational complexity; challenges in real-time performance \\
\addlinespace
Thiruvenkatasamy et al. (2025) & Presents a hierarchical design for distributed data processing and secure ledger storage & Multi-tiered architecture involving edge, fog, and cloud layers for data processing & Increased processing overhead; potential latency due to multi-tier management; limited large-scale deployment assessments \\
\addlinespace
Soy et al. (2025) & Reviews alternatives to traditional PoW for efficient consensus in permissioned blockchain networks & Comparative analysis of PoA, PBFT, and other consensus mechanisms for permissioned blockchains & Trade-offs between decentralization level and energy consumption; performance degradation during high load \\
\addlinespace
Mishra et al. (2023) & Introduces controlled data modification mechanisms while preserving ledger immutability & Redactable blockchain approaches with controlled modification capabilities for IoT networks & Complexity in design; computational overhead may limit adaptability in fast-changing IoT networks \\
\bottomrule
\end{tabularx}
\endgroup
}
\end{table}

\section*{QoS in Blockchain-IoT Systems}
Blockchain-enabled Internet of Things (IoT) systems have increasingly become a critical research focus in applications ranging from secure healthcare to smart agriculture. In smart agriculture, for example, real-time sensing, control and automation are essential for precision farming. However, integrating blockchain technology into IoT deployments introduces a range of QoS challenges, including latency, throughput degradation, energy consumption, and security robustness under dynamic operating conditions. This report examines the recent scholarly contributions (2022--2025) that address QoS issues in blockchain-IoT systems, with an emphasis on advanced consensus mechanisms, queuing theory models and hybrid architectures. We focus particularly on solutions that have been proposed in simulation and empirical experiments, and we critically analyze their limitations. Finally, a mapping of these limitations to potential remediation provided by Hyperledger frameworks is discussed, so that our proposed ``Blockchain-enabled IoT Framework for Smart Agriculture: A CRT-based Parallel Transaction Model with Consensus and QoS Mechanisms'' can build on solid, state-of-the-art insights.

\section*{Mapping Limitations to Hyperledger Solutions}
A recurring limitation in the reviewed literature pertains to scalability and computational overhead linked to consensus mechanisms and real-time transaction processing. Hyperledger Fabric, for instance, provides a modular, permissioned blockchain architecture that enables pluggable consensus modules and configurable endorsement policies, thereby addressing some of these limitations \cite{pajooh2022experimentalperformanceanalysis}. Likewise, models that rely heavily on simulation assumptions and fixed network conditions suffer from reduced reliability under real-world variability; Hyperledger's permissioned framework offers better control over transaction processing and network monitoring, which can be deployed in production-grade environments to support dynamic load balancing. Furthermore, limitations in queuing theory models---namely, their inability to fully account for diverse traffic patterns and unpredictable delay variations---are partially remedied by Hyperledger's integrated smart contract execution platforms that allow real-time analytical adjustments and decentralized resource management. Although no paper currently provides a complete mapping from these limitations to Hyperledger's entire suite of solutions, the architecture's support for flexible, multi-channel communication and its inherent security and scalability benefits make it a promising candidate for addressing many of these technical issues in our proposed smart agriculture framework.

\section*{Integration Into a Smart Agriculture Context}
In the smart agriculture domain, IoT devices are widely deployed for sensor data collection (e.g., soil moisture, temperature, crop health) and must operate with stringent QoS requirements to support real-time monitoring and control. The CRT-based (Concurrent, Reactive, and Transaction-parallel) transaction model proposed in our framework builds on the principles found in the latest research reviewed here. Our design synthesizes two key areas: advanced blockchain transaction management to administer consensus and sidechain selection (inspired by models such as MLSMBQS) and dynamic network scheduling to support low-latency, high throughput processing (as seen in queuing theory applications). Despite their individual limitations---in particular, scalability in blockchain models and rigid assumptions in queuing theory models---merging these approaches in a smart agriculture framework leverages the strong points of each while mitigating weaknesses through real-world deployment and adaptive control mechanisms. The integration strategy envisions the use of Hyperledger Fabric as the backbone of the blockchain infrastructure. This platform's scalability and modular architecture, including its support for private channels and configurable endorsement policies, can help overcome the limitations related to simulation-only results and small-scale deployments found in many studies \cite{pajooh2022experimentalperformanceanalysis}. In addition, by implementing off-chain storage strategies for sensor data, our framework significantly reduces blockchain bloat while sustaining high QoS, a limitation that has been pointed out in hybrid architectural proposals in the literature \cite{haque2024ascalableblockchain}.

\section*{Mapping of Limitations to Hyperledger Solutions}
From our review, the recurring limitations we have identified include: 
\begin{itemize}
\item Scalability and computational overhead in consensus mechanisms that degrade performance under large-scale deployments \cite{agrawal2022mlsmbqsdesignof,haque2024ascalableblockchain}
\item Reliance on simulation environments with fixed traffic assumptions that fail to capture the heterogeneous and dynamic network conditions of real-world smart agriculture settings \cite{habeeb2022dynamicbandwidthslicing,zhang2024ondemandcentralizedresource}
\item Limitations in queuing theory models that do not consider real-time network congestion fluctuations and unpredictable delays \cite{habeeb2022dynamicbandwidthslicing,zhang2024ondemandcentralizedresource}
\item Security and interoperability issues of smart contracts that arise when handling high transaction volumes in decentralized frameworks \cite{gupta2025enhancingblockchainqualityofservice}
\item Network heterogeneity issues that challenge QoS in SDN-IoT scenarios \cite{zafar2023anadvancedstrategy}
\end{itemize}

Hyperledger Fabric's architecture addresses these limitations through its flexible, modular design that supports pluggable consensus protocols, configurable smart contract deployment, and dynamic blockchain channel management. In particular, Hyperledger's mechanisms for off-chain storage integration and private channel creation help mitigate the scalability and latency issues observed in many simulation-based studies, while its permissioned network model provides robust security and improved performance in real-world heterogeneous deployments \cite{pajooh2022experimentalperformanceanalysis}. Although detailed mappings require further experimental evaluation, the potential of Hyperledger to resolve these limitations makes it an attractive option for our proposed framework.

\section*{Integration for Smart Agriculture: Proposed Framework}
The proposed ``Blockchain-enabled IoT Framework for Smart Agriculture'' builds on insights derived from the reviewed literature and leverages a CRT-based parallel transaction model for robust consensus and improved QoS. Our design incorporates the following key components:
\begin{itemize}
\item A blockchain layer based on Hyperledger Fabric that supports parallel transaction processing using private channels and modular consensus, ensuring low network latency and high throughput even as sensor data volumes grow
\item A dynamic queuing mechanism that integrates bandwidth slicing and resource allocation models to prioritize time-critical agricultural sensor data, leveraging machine learning for adaptive resource management
\item A hybrid on-chain/off-chain architecture wherein high-volume data (e.g., continuous sensor streams) are stored off-chain while critical metadata and transaction logs are committed to the blockchain, thereby maintaining integrity and reducing blockchain bloat
\item Integration of SDN techniques to manage heterogeneous IoT connectivity, ensuring that controller response times are minimized and network scheduling is optimized for real-time agricultural operations
\item A comprehensive monitoring module that collects data on QoS metrics (e.g., throughput, latency, and energy consumption) and uses real-time analytics to adjust system parameters dynamically
\end{itemize}

This integration of blockchain, queuing theory, and SDN-based scheduling is tailored to meet the stringent QoS requirements of modern smart agriculture---in which timely, secure, and accurate sensor data directly impact crop management and yield optimization. The design directly addresses the limitations observed in current research: scalability and limited real-world validation (as seen in simulation-only studies), and the challenges of achieving low latency amid dynamic network loads. The use of Hyperledger Fabric is especially critical, given its proven performance in experimental studies \cite{pajooh2022experimentalperformanceanalysis} and the advanced features that allow flexible consensus management.

\section{Future Research Directions and Open Challenges}\label{sec:future}
While significant progress has been achieved, notable gaps and challenges remain in achieving fully scalable, interoperable, and real-time blockchain-enabled IoT frameworks for smart agriculture. First, although Layer-2 scaling, sharding, and specialized consensus algorithms like CRPBFT have demonstrably increased throughput and reduced latency, a universally scalable solution that can handle millions of transactions per day without compromising security or raising energy costs remains elusive \cite{huang2025digitaltraceabilityin, irfan2025aniotdrivensmart}. In many rural settings, economic constraints may result in blockchain deployment costs consuming substantial portions of annual income for smallholder farmers; therefore, cost-effective models are urgently needed \cite{irfan2025aniotdrivensmart, irfan2025aniotdrivensmart}.

Second, interoperability challenges persist because the integration of heterogeneous IoT devices from various manufacturers with disparate data structures continues to degrade the quality and context of sensor data during cross-chain transfers. Although semantic frameworks using RDF/OWL have shown improvements in maintaining contextual integrity, there is still a lack of unified global standards to ensure seamless interoperability across diverse platforms \cite{irfan2025aniotdrivensmart, huang2025digitaltraceabilityin}.

Third, real-time processing remains a delicate issue when blockchain's inherent confirmation delays conflict with the need for immediate decision-making in high-stakes agricultural operations. While edge-AI models and federated learning can reduce local inference times to sub-50 millisecond levels, ensuring that the overall system continues to provide timely responses---especially when combined with the slower pace of global consensus---requires further innovation \cite{huang2025digitaltraceabilityin, huang2025digitaltraceabilityin}.

The proposed CRT-based parallel transaction model, by partitioning the transaction load across multiple microchains and integrating dynamic QoS mechanisms, offers a promising avenue to address these trade-offs concurrently. Future research should focus on comprehensive simulation studies that incorporate realistic network conditions and device-specific characteristics. Field trials involving diverse operational environments and large-scale sensor deployments will be crucial to validate the model's efficiency, energy consumption, and overall economic viability \cite{thiruvenkatasamy2025anonlinetool, thiruvenkatasamy2025anonlinetool}.

Moreover, interdisciplinary collaboration---which brings together expertise from distributed ledger technology, IoT engineering, agricultural science, and economics---is essential to refine technical models and align them with practical farming needs. Future work should also explore integration with emerging technologies such as 5G networks, AI-driven robotics, and renewable energy--powered edge nodes to create a more resilient ecosystem for smart agriculture \cite{thiruvenkatasamy2025anonlinetool, alazzai2024smartagriculturesolutions}.

The rapid evolution of smart agriculture has spurred the development of innovative data management and security paradigms that overcome the limitations of conventional centralized systems. In an era where Internet of Things (IoT) devices are widely deployed across farms for monitoring crop health, irrigation levels, and environmental parameters, ensuring data integrity, traceability, and prompt decision-making is of paramount importance. Blockchain technology, with its decentralized ledger, immutable records, and flexible consensus mechanisms, promises to revolutionize smart agriculture by providing ``farm-to-fork'' transparency and robust security guarantees. Recent scholarly works \cite{ali2022blockchainenabledarchitecture, aliyu2023blockchainbasedsmartfarm} have explored blockchain-enabled IoT frameworks that integrate lightweight consensus selection, parallel transaction processing based on the Chinese Remainder Theorem (CRT), and quality-of-service (QoS) enhancements to support the high-volume, heterogeneous data environments typical in agriculture.  This report reviews recent literature published roughly between 2022 and 2025 that investigates blockchain architectures for smart agriculture. The analysis highlights different methodologies used for integrating blockchain with IoT networks and examines two major focus areas: (i) the development of adaptive consensus mechanisms---including selective consensus, dynamic protocol switching, and CRT-based parallel transaction models---and (ii) the use of blockchain for food traceability across agri-food supply chains. In addition, emerging frameworks that incorporate advanced metaheuristics into blockchain systems are discussed, and limitations common to many studies are mapped to potential solutions provided by enterprise-grade platforms such as Hyperledger Fabric. Hyperledger's modular design, channel-based parallel transaction processing, pluggable consensus protocols, and lightweight client support collectively address many of the critical challenges reported in the reviewed literature \cite{ali2022blockchainenabledarchitecture}.  



%%%%%%%%%%%%%%%%%%%%%%%%%%%%%%%%%%%%%%%%%%%%%%%%%%%%%%%%%%%%
% PART VI – Conclusion and References
%%%%%%%%%%%%%%%%%%%%%%%%%%%%%%%%%%%%%%%%%%%%%%%%%%%%%%%%%%%%
\part*{Part VI – Conclusion and References}\addcontentsline{toc}{part}{Part VI – Conclusion and References}

\section{Conclusion}\label{sec:conclusion}
Blockchain-enabled IoT frameworks stand at the forefront of transforming smart agriculture by offering significant improvements in data security, traceability, and operational efficiency. The literature reviewed herein demonstrates that adaptive consensus mechanisms, CRT-based parallel processing techniques, and metaheuristic optimizations can collectively enhance the performance and scalability of blockchain systems in agriculture. Nevertheless, persistent challenges remain—scalability limits, heterogeneous data integration, computational constraints of IoT devices, and real-world validation of novel consensus protocols. The mapping exercise presented in this report shows that these limitations can be strategically addressed by leveraging enterprise-grade platforms such as Hyperledger Fabric, which offer modular consensus plug-ins, lightweight client support, channel-based throughput improvement, and robust security and identity management features.

Future research should prioritize large-scale empirical validations of CRT-based parallel transaction models with dynamic consensus adaptation and focus on integrating advanced methodologies such as machine learning for predictive security measures. Bridging the gap between controlled laboratory prototypes and scalable, deployable field systems will be essential for realizing the full potential of blockchain-enabled smart agriculture. In so doing, it will be possible to establish a new era of agricultural management that ensures end-to-end traceability of food supply chains, resilient security for distributed sensor networks, and optimized use of resources in precision crop monitoring.

This report has provided a comprehensive review of recent scholarly work on blockchain-enabled IoT frameworks for smart agriculture from 2022 to 2025, structured into four key areas: blockchain in smart agriculture, IoT architectures, consensus mechanisms and performance, and QoS in blockchain-IoT systems. Through the synthesis of more than 20 studies and the creation of detailed comparison tables, we have identified critical limitations such as scalability challenges, inefficient consensus protocols for resource-constrained environments, and QoS bottlenecks. In mapping these limitations to potential Hyperledger solutions, we note that Hyperledger Fabric and related frameworks offer powerful tools for optimizing consensus mechanisms, enhancing interoperability, and bolstering security and performance in decentralized networks. Future research directions should pursue real-world validation, adaptive consensus tuning, and tighter integration between blockchain, IoT, and edge analytics, ultimately advancing the vision of highly efficient, transparent, and secure smart agriculture systems.

The convergence of IoT, edge computing, AI, and blockchain technologies represents a transformative opportunity for the agricultural sector. This enhanced literature review has synthesized recent research contributions \cite{atalla2023iotenabledprecisionagriculture, akhter2022precisionagricultureusing, atalla2023iotenabledprecisionagriculture, bayih2022utilizationofinternet, ouafiq2022datamanagementand} and identified key limitations, including scalability challenges, interoperability gaps, energy inefficiencies, and data security concerns. By mapping these limitations to potential solutions provided by Hyperledger’s blockchain framework—such as decentralized consensus models, secure smart contracts, and standardized data integration protocols—the report underlines a promising pathway toward developing a robust, secure, and scalable IoT system for smart agriculture. Future work must focus on real-world deployments, further optimization of energy usage, and extensive field validations to fully realize the benefits of blockchain-enabled IoT architectures in precision agriculture.

In summary, recent scholarly work on QoS in blockchain-IoT systems emphasizes the benefits of adopting machine learning techniques, lightweight consensus mechanisms, and advanced network scheduling strategies to deliver improved throughput, lower latency, and enhanced energy efficiency. Nonetheless, common limitations persist, particularly with respect to scalability under real-world conditions and the reliability of simulation-based evaluations. Hyperledger Fabric offers a powerful alternative thanks to its modular design, flexible consensus modules, and support for secure smart contracts, making it a promising platform for addressing these limitations in smart agriculture contexts. Our proposed framework leverages these insights to provide a robust, scalable, and secure blockchain-enabled IoT solution for smart agriculture that meets the demanding QoS requirements of next-generation agricultural applications. Future work will focus on real-world validation with extensive field trials and the integration of additional adaptive control mechanisms to continuously optimize performance across heterogeneous smart agriculture networks.

The enhanced literature review clearly indicates that while significant progress has been made in improving QoS in blockchain-enabled IoT systems—with advancements in ML-driven blockchain management, lightweight consensus mechanisms, queuing theory for dynamic scheduling, and SDN-based network heterogeneity management—several practical limitations remain. These limitations, predominantly related to scalability, simulation-based evaluations, and stringent network conditions, are precisely the areas that Hyperledger Fabric's modular and secure architecture can address. Our proposed framework for smart agriculture will leverage these insights to achieve robust, scalable, and secure performance, ultimately enabling precise and efficient IoT-based agricultural operations.

This comprehensive review, alongside the state-of-the-art comparison table and the mapping of limitations to Hyperledger's solutions, provides a solid foundation upon which to develop and validate our CRT-based parallel transaction model for smart agriculture.

This comprehensive review has synthesized findings across multiple domains including blockchain applications, IoT architectures, consensus mechanisms, and QoS research in smart agriculture. Revisiting the goals outlined in the Introduction, we conclude that blockchain-enabled IoT frameworks are evolving rapidly to address the dual challenges of securing data against sophisticated cyber-attacks while preserving user privacy in decentralized networks. The integration of permissioned blockchain architectures with advanced cryptographic methods, federated learning, and real-time automated alert systems offers significant enhancements in data integrity, privacy, and reliability for agricultural applications.

The research demonstrates that blockchain-enabled IoT frameworks hold transformative potential for smart agriculture by simultaneously addressing critical issues of energy management, usability, and supply chain traceability. Energy-efficient architectures employing lightweight consensus protocols and low-power communication technologies have shown promise in reducing the operational energy footprint of agricultural IoT networks \cite{munaganuri2025designofan, bapatla2023easychainaniotfriendly}. Furthermore, the integration of blockchain technology with IoT offers substantial benefits for enhancing data transparency, security, and operational efficiency across the agricultural supply chain.

Despite these advancements, significant challenges persist. Practical deployment faces obstacles related to scalability, computational overhead, and energy consumption. Blockchain adoption by farmers is strongly influenced by usability factors, while traceability systems remain limited by partial process integration, interoperability challenges, and scalability issues \cite{akella2023asystematicreview, mwewa2024blockchaintechnologya}. The high volume of sensor data, disparate device ecosystems, and need for real-time responsiveness continue to pose technical and economic hurdles \cite{huang2025digitaltraceabilityin, irfan2025aniotdrivensmart}. 

The proposed CRT-based parallel transaction model with integrated consensus and QoS mechanisms represents a novel approach to address these challenges. By partitioning transactions into parallel streams and dynamically prioritizing critical sensor events, this model demonstrates potential to significantly reduce processing latency and increase throughput while ensuring robust security and interoperability across heterogeneous systems. This integrated approach aims to maximize energy and computational efficiency while ensuring user-friendly interfaces and full-stack supply chain integration \cite{thiruvenkatasamy2025anonlinetool, singh2025blockchainandflbased}.

Future research should focus on developing lightweight, adaptive frameworks that harmonize security, privacy, and reliability while accommodating resource constraints of rural farming environments. Validation through extensive simulation, rigorous field testing, and interdisciplinary research is required to optimize the proposed model in real-world agricultural settings. Continued collaborative efforts among researchers, industry stakeholders, and policymakers will be critical in developing standardized protocols and economically viable models.

By addressing these gaps and pursuing these research directions, blockchain-enabled IoT systems can provide a robust foundation for sustainable, efficient, and secure smart agriculture, as envisioned in the project "Blockchain-enabled IoT Framework for Smart Agriculture: A CRT-based Parallel Transaction Model with Consensus and QoS Mechanisms" \cite{aliyu2023blockchainbasedsmartfarm, daund2025designofan}. Continued innovation in mathematical partitioning techniques, adaptive consensus algorithms, and QoS integration will bridge the remaining gaps, ultimately transforming modern agriculture through timely, data-driven, and secure operational frameworks.


The proposed CRT-based parallel transaction architecture and its multi-tiered data processing flow are illustrated; similar multi-tier architectures have been detailed in previous studies~\cite{thiruvenkatasamy2025anonlinetool}. Inline IEEE-style citations throughout this report, such as~\cite{huang2025digitaltraceabilityin, irfan2025aniotdrivensmart} and~\cite{huang2025digitaltraceabilityin, huang2025digitaltraceabilityin}, reflect the extensive research contributions from 2022 to 2025.

In conclusion, while significant technical challenges remain with respect to scalability, interoperability, and real-time processing in blockchain-enabled IoT frameworks for smart agriculture, the innovative CRT-based parallel transaction model with adaptive consensus and QoS mechanisms represents a robust, flexible, and economically viable solution. This approach has the potential to revolutionize precision agriculture by efficiently managing vast streams of sensor data, harmonizing disparate device protocols, and enabling rapid, responsive decision-making, ultimately supporting sustainable food supply chains and empowering smallholder farmers on a global scale \cite{huang2025digitaltraceabilityin, thiruvenkatasamy2025anonlinetool, huang2025digitaltraceabilityin, irfan2025aniotdrivensmart}.

Future work must focus on extensive field deployments, standardized interoperability frameworks, and continuous interdisciplinary collaboration to ensure that these advanced systems not only meet technical requirements but also align with the economic and practical constraints of modern agriculture. The promising early results and theoretical underpinnings described in this report offer a clear path forward for realizing fully integrated, scalable, and real-time blockchain-enabled IoT frameworks that can transform the agricultural landscape over the coming years \cite{thiruvenkatasamy2025anonlinetool, alazzai2024smartagriculturesolutions}.

\section{References}\label{sec:references}
% Place your bibliography command here if compiling this file directly, e.g.:
% \bibliographystyle{IEEEtran}  % or your style
% \bibliography{out/references}

\clearpage
\bibliographystyle{IEEEtran}
\bibliography{references}  % <-- uses references.bib in the same directory

\end{document}
