\documentclass{article}
\usepackage[utf8]{inputenc}
\usepackage{hyperref}

\title{Hyperledger-Based IoT Data Integrity: A Scientific Overview}
\author{Author Name}
\date{\today}

\begin{document}

\maketitle

\begin{abstract}
This paper provides an overview of the Hyperledger-based system for ensuring data integrity in IoT networks. It outlines the architecture, discusses the role of blockchain technology, and reviews the key components implemented in the accompanying repository.
\end{abstract}

\section{Introduction}
Hyperledger frameworks offer modular tools for deploying blockchain solutions. This document describes how the provided repository leverages these frameworks to build a secure and scalable infrastructure for IoT data management.

\section{System Architecture}
Describe the system components such as sensor nodes, edge processors, and the blockchain network. Explain how they interact to preserve data integrity and enable auditability.

\section{Implementation Details}
Summarize the implementation found in the repository, including smart contracts, network configuration, and data flow between devices and the ledger.

\section{Evaluation}
Discuss potential evaluation metrics for performance, security, and reliability. Outline experiments or simulations to validate the system.

\section{Conclusion}
Provide concluding remarks and future work directions, such as integrating additional security mechanisms or scaling the network for real-world deployments.

\bibliographystyle{plain}
\bibliography{references}

\end{document}
